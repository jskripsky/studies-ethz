\documentclass{article}
\usepackage{latexsym}
\usepackage{amssymb}
\usepackage{epsfig}
\usepackage{amsmath}
\usepackage{amsthm}
\usepackage{amstext}
\usepackage{amscd}


\newtheorem{defi}{Definition}
\newtheorem{theo}{Theorem}
\newtheorem{lem}[theo]{Lemma}
\newtheorem{cor}[theo]{Corollary}
\newtheorem{rem}{Remark}
\newtheorem{con}{Conjecture}
\newtheorem{prop}[theo]{Proposition}

\def\u{\uparrow}
\def\d{\downarrow}
\def\F{{F}}
\def\D{{D}}
\def\I{{\cal I}}
\def\half{\frac{1}{2}}
\def\ket#1{|\,#1\,\rangle}
\def\bra#1{\langle\, #1\,|}
\def\braket#1#2{\langle\, #1\,|\,#2\,\rangle}

\newcommand{\textcirc}{% circle of reasonable size 
  \setlength{\unitlength}{1.5ex}%
  \begin{picture}(1.1,1)(-.55,-.55)%
    \put(0,0){\circle{1.1}}%
  \end{picture}}
\newcommand{\defend}{\hspace*{\fill} $\textcirc$}
\newcommand{\exend}{\hspace*{\fill} $\diamondsuit$}
\newcommand{\Tr}{{\rm Tr}}
\newcommand{\HA}{{\cal H}_{{A}}}
\newcommand{\HB}{{\cal H}_{{B}}}
\newcommand{\HE}{{\cal H}_{{E}}}
\newcommand{\oz}{\overline{z}}
\newcommand{\prob}{{\rm Prob\,}}
\newcommand{\cd}{{D}}
\newcommand{\varl}{l}
\newcommand{\ul}{\underline}
\newcommand{\noi}{\noindent}
\newcommand{\cancel}[1]{}
\newcommand{\canpro}[1]{}
\newcommand{\vs}[1]{\vspace{#1mm}}
\newcommand{\vsk}{\vspace{3mm}}
\newcommand{\vsl}{\vspace{7mm}}
\newcommand{\ind}{\rule{7mm}{0mm}}
\newcommand{\rh}[1]{\rule{#1 mm}{0mm}}
\newcommand{\rv}[1]{\rule{0mm}{#1 mm}}
\newcommand{\rz}{\rule{0mm}{0mm}}
\newcommand{\pp}{,\ldots,}
\newcommand{\be}{\beta}
\newcommand{\ee}{\end{equation}}
\newcommand{\bea}{\begin{eqnarray}}
\newcommand{\beq}{\begin{equation}}
\newcommand{\eea}{\end{eqnarray}}
\newcommand{\beas}{\begin{eqnarray*}}
\newcommand{\eeas}{\end{eqnarray*}}
\newcommand{\bece}{\begin{center}}
\newcommand{\ence}{\end{center}}
\newcommand{\beit}{\begin{itemize}}
\newcommand{\enit}{\end{itemize}}
\newcommand{\nli}{\newline}
\newcommand{\hyp}{\hyphenation}
%\newcommand{\mod}{{\rm mod\ }}
\newcommand{\co}{{\cal O}}
\newcommand{\lgr}{\langle g\rangle}
%\newcommand{\proof}{{\it Proof. }}
\newcommand{\proofend}{\hspace*{\fill} $\Box$\\ \ \\}
\newcommand{\pe}{\hspace*{\fill} $\Box$\\ \ \\}
\newcommand{\peon}{\hspace*{\fill} $\Box$\\}
\newcommand{\DHO}{{\rm DHO}}
\newcommand{\RSA}{{\rm RSA}}
\newcommand{\ord}{{\rm ord}}
\newcommand{\OC}{{\rm OC}}
\newcommand{\OCs}{{\rm OCs}}
\newcommand{\fp}{{\bf F}_p}
\newcommand{\fq}{{\bf F}_q}
\newcommand{\CE}{{E}}
\newcommand{\zn}{({\bf Z}/{n{\bf Z}})^*}
\newcommand{\znn}{{\bf Z}/{n{\bf Z}}}
\newcommand{\z}{{\bf Z}}
\newcommand{\zz}{{\bf Z}}
\newcommand{\zznn}{{\bf Z}_n}
\newcommand{\al}{\alpha}
\newcommand{\fqN}{{\bf F}_{q^N}}
\newcommand{\fpn}{{\bf F}_{p^n}}
\newcommand{\fpk}{{\bf F}_{p^k}}
\newcommand{\fpi}{{\bf F}_{p^i}}
\newcommand{\fpg}{{\bf F}_{p^g}}
\newcommand{\fpm}{{\bf F}_{p^m}}
\newcommand{\fqn}{{\bf F}_{q^n}}
\newcommand{\fqi}{{\bf F}_{q^i}}
\newcommand{\fpN}{{\bf F}_{p^N}}
\newcommand{\ep}{\varepsilon}
\newcommand{\de}{\delta}
\newcommand{\get}{\tilde{g}}
\newcommand{\vp}{\varphi}
\newcommand{\hterm}[1]{\mbox{{\rm hterm(}$#1${\rm )}}}
\newcommand{\MM}[1]{\mbox{{\rm M(}$#1${\rm )}}}
\newcommand{\hcoeff}[1]{\mbox{{\rm hcoeff(}$#1${\rm )}}}
\newcommand{\fei}[2]{\mbox{{\rm #1(}$#1${\rm )}}}
\newcommand{\DD}{{\bf D}}
\newcommand{\DO}{{\bf D}_0}
\newcommand{\PP}{{\bf P}}
\newcommand{\JJ}{{\bf J}}
\newcommand{\di}[2]{\mbox{{\rm div(}$#1,#2${\rm )}}}
\newcommand{\spoly}[2]{\mbox{{\rm s-poly(}$#1,#2${\rm )}}}
\newcommand{\lcm}[2]{\mbox{{\rm lcm(}$#1,#2${\rm )}}}
\newcommand{\fzw}[3]{\mbox{{\rm #1(}$#2,#3${\rm )}}}
\newcommand{\gcdd}[2]{\mbox{{\rm gcd(}$#1,#2${\rm )}}}
\newcommand{\OO}{{\bf 0}}
\newcommand{\Op}{{\bf 0}_p}
\newcommand{\Oq}{{\bf 0}_q}
%\newcommand{\al}{\alpha}
%\newcommand{\de}{\delta}
%\newcommand{\ep}{\epsilon}
\newcommand{\ga}{\gamma}
\newcommand{\ra}{\rightarrow}
\newcommand{\lra}{\longrightarrow}
\newcommand{\Lra}{\Longrightarrow}
\newcommand{\llra}{\longleftrightarrow}
\newcommand{\Llra}{\Longleftrightarrow}
\newcommand{\toinf}{\rightarrow\infty}
\newcommand{\pacc}{p_{a,N}}
\newcommand{\perr}{p_{\mbox{\scriptsize err}}}
\newcommand{\bnp}{\mbox{\scriptsize Bin}(N,p)}
\newcommand{\ww}{\tilde{w}}
\newcommand{\uq}{U}
\newcommand{\zq}{V}
\newcommand{\tx}{\tilde{X}}
\newcommand{\ty}{\tilde{Y}}
\newcommand{\tz}{\tilde{Z}}
\newcommand{\tr}{\tilde{R}}
%\newcommand{\op}{\oplus}
\newcommand{\op}{\oplus}
\newcommand{\oy}{\overline{y}}
\newcommand{\OX}{\overline{X}}
\newcommand{\OR}{\overline{R}}
\newcommand{\OCA}{\overline{C_A}}
\newcommand{\OCB}{\overline{C_B}}
\newcommand{\OCE}{\overline{C_E}}
\newcommand{\OY}{\overline{Y}}
\newcommand{\OZ}{\overline{Z}}
\newcommand{\Oz}{\overline{z}}
\newcommand{\obe}{\overline{\beta}}
\newcommand{\oga}{\overline{\gamma}}
\newcommand{\cx}{{\cal X}}
\newcommand{\hcx}{\hat{{\cal X}}}
\newcommand{\hcy}{\hat{{\cal Y}}}
\newcommand{\cy}{{\cal Y}}
\newcommand{\cz}{{\cal Z}}
\newcommand{\CX}{{\cal X}}
\newcommand{\CY}{{\cal Y}}
\newcommand{\CZ}{{\cal Z}}
\newcommand{\hX}{\hat{ X}}
\newcommand{\hY}{\hat{ Y}}
\newcommand{\hZ}{\hat{ Z}}
\newcommand{\hz}{\hat{ z}}
\newcommand{\gd}{\Delta}
\newcommand{\dq}{\overline{\gd}}
\newcommand{\cer}{{\cal R}}
\newcommand{\da}{\downarrow}
\newcommand{\px}{\overline{P_X}}
\newcommand{\py}{\overline{P_Y}}
\newcommand{\din}{d_{\scriptsize ind}(P_{XY})}
\newcommand{\ida}{I(X;Y\hspace{-1.3mm}\downarrow\hspace{-0.8mm} Z)}
\newcommand{\idazwei}{I(\OX;Y\hspace{-1.3mm}\downarrow\hspace{-0.8mm} Z)}
\newcommand{\idadrei}{I(X;\OY\hspace{-1.3mm}\downarrow\hspace{-0.8mm} Z)}
\newcommand{\penf}{\vspace{-1cm}\hspace*{\fill} $\Box$\\ \ \\}
\newcommand{\peinf}{\hspace*{\fill} \Box\\ \ \\}

\def\half{\frac{1}{2}}
 \def\<{\langle}
 \def\>{\rangle}
 \def\up{\uparrow}
 \def\down{\downarrow}
 \def\lefta{\leftarrow}
 \def\righta{\rightarrow}
 \def\opone{\leavevmode\hbox{\small1\kern-3.8pt\normalsize1}}
 \def\H{{\cal H}}
 \def\A{{\cal A}}
 \def\B{{\cal B}}
 \def\E{{\cal E}}
 \def\z{{\bar z}}
 \newcommand{\complex}{{\kern .1em {\raise .47ex\hbox {$\scriptscriptstyle
 |$}}\kern -.4em {\rm C}}}
 \newcommand{\real}{{{\rm I} \kern -.19em {\rm R}}}
\newcommand{\eeq}{\end{equation}}
 \newcommand{\beqa}{\begin{eqnarray}}
 \newcommand{\eeqa}{\end{eqnarray}} 


\begin{document}

\title{
\bf Linking Classical and Quantum Key Agreement:
 Is There a Classical Analog to Bound Entanglement?
\footnote{Some of the results of this paper already  appeared
in~\cite{giswol99}, \cite{giswol00}, and~\cite{grw00}.}}



\author{Nicolas Gisin\, \footnote{Group 
of Applied Physics, University of Geneva,
CH-1211 Geneva, Switzerland. E-mail: Nicolas.Gisin@physics.unige.ch}\qquad
Renato Renner\, \footnote{Computer Science Department, Swiss Federal 
Institute of Technology (ETH Z\"urich), CH-8092 Zurich, Switzerland.
E-mail: renner@inf.ethz.ch}\qquad Stefan Wolf\, \footnote{Centre for Applied 
Cryptographic Research,
Department of Combinatorics and Optimization,
University of Waterloo, 
Waterloo, ON N2L 3G1, Canada. E-mail:
swolf@cacr.math.uwaterloo.ca}}


\date{}







\maketitle




\begin{abstract}
\noi
After carrying out a protocol for quantum key agreement over a noisy quantum
channel, the parties Alice and Bob must process the raw key in order to 
end up with identical keys
about which 
the adversary has virtually no 
 information. 
In principle, both classical and quantum protocols can be used for this 
processing. It is a natural question which type of protocols is more 
powerful. We show that   the limits of tolerable 
noise are identical for classical and quantum protocols in many cases. 
More specifically, we prove that a quantum state between two parties
is entangled if and only if the classical random variables resulting 
from optimal measurements provide some mutual classical information 
between the parties.  
In addition, we present evidence which strongly suggests that the 
potentials
of classical and of quantum protocols are equal in every situation.
An important consequence, in the purely classical regime, of such a 
correspondence would be the existence of a classical counterpart
of so-called bound entanglement, namely 
``bound information'' that 
cannot be used for generating a secret key by any protocol. 
This stands in  contrast to what was previously believed.
The studied connection between the classical and
quantum protocols makes it natural to conjecture that (classical and
quantum) distillability is possible only if single-copy distillability is
already possible.
\\ \ \\
{\bf Keywords.} 
Secret-key agreement,  intrinsic  information,
 secret-key rate, purification, entanglement.
\end{abstract}


\newpage

\section{Introduction}

In modern cryptography there are mainly two security paradigms, namely 
computational and information-theoretic security. The latter is sometimes also 
called unconditional security. Computational security is based on the 
assumed hardness of certain computational problems (e.g., the integer-factoring
or  dis\-crete-logarithm problems). However, since a computationally 
sufficiently powerful adversary can solve any computational
 problem, hence break any 
such system, and because no useful general lower bounds are known in complexity 
theory, computational
 security is always conditional and, in addition to this, in danger by  progress
in the theory of efficient algorithms as well as in hardware engineering 
(e.g., quantum computing). 
Information-theoretic security on the other hand is based on probability 
theory and on the fact that an adversary's information is limited.
Such a limitation can for instance come from noise in communication 
channels or from the laws of quantum mechanics. 

Many different cryptographic settings 
based on  noisy channels have been 
described and analyzed. Examples are  Wyner's wire-tap channel~\cite{wyner75},
Csisz\'ar and K\"orner's broadcast channel~\cite{csikor78}, or 
Maurer's model of  key agreement from joint randomness~\cite{ka},\,
 \cite{ittrans}. 


Quantum cryptography on the other hand lies in the intersection of two of the 
major scientific achievements of the 20th century, namely quantum physics and 
information theory. Various protocols for so-called quantum
key agreement have been proposed  (e.g., \cite{BB84},\, \cite{ekert}), 
and
the possibility and impossibility of 
such key agreement in different settings has been 
studied by many authors. 



The goal of this paper is to derive parallels between classical and 
quantum key agreement
and thus to show that the two paradigms are 
more closely related  than previously recognized.
These connections allow for  investigating questions and  solving
open problems of purely classical information theory with quantum-mechanic
methods. 
One of the possible consequences is that, in contrast to what was 
previously believed, there exists a classical counterpart to so-called
{\em bound entanglement\/} (i.e., entanglement that cannot be purified by any quantum
protocol), namely mutual  information between Alice and Bob 
which they cannot use for generating a secret key by any classical protocol.



The outline of this paper is as follows. In Section~\ref{models} we introduce 
the classical (Section~\ref{classical}) and quantum (Section~\ref{quantum})
models 
of information-theoretic key agreement
and the crucial  concepts and quantities, such as  secret-key 
rate and intrinsic information on one side, and measurements,
entanglement, and quantum
purification on the other.
In Section~\ref{linking} we show the mentioned links between these two 
models, more precisely, between entanglement and intrinsic information
(Section~\ref{linkingone}) as well as between quantum purification
and the secret-key rate (Section~\ref{prot}).
We illustrate the statements and their consequences with a number of examples
(Sections~\ref{exampleseins} and~\ref{exampleszwei}). 
In Section~\ref{secbii} we define and characterize the classical 
counterpart of bound entanglement, called bound intrinsic information. 
We show that 
not only   
problems  in classical information theory can be 
addressed by quantum-mechanical methods, but that the inverse is also  true:
In Section~\ref{linkingmeasure}
we propose a new measure for entanglement  
based on the intrinsic information measure.



The results of Section~3 already appeared in~\cite{giswol99} 
and~\cite{giswol00}. Proposition~\ref{prop1} in Section~4 was 
proved in~\cite{grw00}, whereas the other  results in Section~4 
have not been  published previously. 









\section{Unconditionally Secure Key Agreement}
\label{models}


Shannon~\cite{shannon} has defined an encryption scheme to be 
{\em perfectly secret\/}
if the ciphertext does not reveal 
any information about the encrypted message. Such a system 
is  unconditionally secure with respect to a ciphertext-only 
attack; in particular, an exhaustive search over the key space is 
of no help for finding the cleartext. Shannon 
proved in the same paper that, unfortunately, such a high level of secrecy has
its price: it is, roughly spoken, only possible between parties
who share an information-theoretically secure key that is at least 
as long as the message to be encrypted. The so-called {\em one-time
pad\/}~\cite{vernam26}, a computationally very simple encryption 
that just bit-wisely XORs the key to the message, on the other 
hand shows that perfectly secure encryption is possible between parties
who do share a key of that length. Since we assume that insecure 
channels are always available, the one-time pad  reduces the problem
of information-theoretically secure encryption to 
information-theoretically secure {\em key agreement}, which we will 
consider in the following.




\subsection{Information-Theoretic Key Agreement from Classical and Quantum 
Information}

We assume 
 that two parties Alice and Bob, who  are connected by an authentic 
but otherwise completely insecure channel,  are willing to generate a
secret key. More precisely,
Alice and Bob want to compute, after some rounds of communication
(where the random variable $C$ summarizes the communication 
carried out over the public channel), strings $S_A$ and $S_B$, respectively,
with the property that they are most likely
both equal to a uniformly distributed 
string $S$ about which the adversary Eve has virtually no information.
More precisely, 
\begin{equation}\label{iss}
\prob[S_A=S_B=S]\geq 1-\ep\ ,\ \ H(S)=\log_2|{\cal S}|
\ ,\ \ \mbox{and}\ \ I(S;C)<\ep
\end{equation}
(where ${\cal S}$ is the range of $S$ and $|{\cal S}|$ is its cardinality)
should hold for some small~$\ep$. 
Note that the security condition in  (\ref{iss}) is 
information-theoretic (sometimes also called unconditional): Even 
an adversary with unlimited computer power must be unable to 
obtain  useful information. In contrast to this, 
the Diffie-Hellman protocol~\cite{difhel76}
for instance achieves the goal of key agreement by insecure communication
only with respect to  computationally bounded adversaries. 

It is a straight-forward generalization of Shannon's mentioned
impossibility result that information-theoretic 
secrecy cannot be generated in this setting, i.e., 
 from authenticity only: Public-key 
systems are never unconditionally secure.
Hence we have to assume some additional structure in the initial
setting, for 
instance some pieces of information given to Alice and Bob (and also Eve),
respectively.





\subsection{Classical Information}\label{classical}

The general case where this information given to the three parties initially 
consists of the outcomes of some random experiment has been studied 
intensively~\cite{ka},\, 
\cite{ittrans},\, \cite{diss}. Here, it is assumed that 
Alice, Bob, and Eve have access to realizations of random
variables $X$, $Y$, and $Z$, respectively, 
jointly distributed according to $P_{XYZ}$.
A special case is 
when all the parties receive noisy versions of a (binary)
signal broadcast 
by some information source. 

It was shown that if the setting is modified
this way
(where the secrecy condition in (\ref{iss}) must be replaced by 
$I(S;CZ)<\ep$), then secret-key agreement is often  possible. Shannon's 
pessimistic result now generalizes to the statement that the size 
of the resulting secret key $S$ cannot  exceed the quantity
\[
I(X;Y\downarrow Z):=\min_{XY\rightarrow Z\rightarrow \overline{Z}} I(X;Y|\OZ)
\]
(where the minimum is taken over all Markov chains 
$XY\rightarrow Z\rightarrow \overline{Z}$)
which was defined in~\cite{ittrans} as the {\em intrinsic
conditional information between $X$ and $Y$, given $Z$}.

In the special case where the parties' initial information consists of the 
outcomes of many independent repetitions of the same random experiment
given
by $P_{XYZ}$
(i.e., Alice knows $X^N:=[X_1,X_2,\ldots,X_N]$, and similarly for 
Bob and Eve),
the {\em secret-key rate\/} $S(X;Y||Z)$ was defined as the 
maximal
 key-generation rate (measured with respect 
to the number of required  realizations of $P_{XYZ}$)  that is 
 asymptotically achievable (for $N\rightarrow\infty$). The above-mentioned 
result then implies
\[
S(X;Y||Z)\leq \ida\ ,
\]
and it was conjectured that intrinsic information can always be distilled 
into a secret key, i.e., that $\ida>0$ implies $S(X;Y||Z)>0$~\cite{ittrans},\, 
\cite{diss}. 
This conjecture was supported by some evidence given in~\cite{ittrans};
however,
it is the objective of this paper to give much stronger evidence 
for the opposite, i.e., that there exist types of intrinsic information
{\em not\/} allowing for secret-key agreement. The motivation for the 
corresponding considerations comes from quantum mechanics or,  more 
precisely, from the concept of {\em bound entanglement\/} in 
quantum information theory.





\subsection{Quantum Information}\label{quantum}

When considering the model where certain pieces of information 
are given initially to the involved parties, it is a natural 
question where this information comes from. According to Landauer,
information is always physical and hence ultimately quantum 
mechanical~\cite{landau98},\, \cite{landau96}. 
Thus the random variables could come from measuring 
a certain quantum state $|\Psi\rangle$. In this case however
it seems to be overly restrictive to force Alice and Bob to measure 
their quantum systems right at the beginning of the key-agreement 
process. It is possibly  advantageous for them to 
carry out a protocol first (using classical communication and local
quantum operations on their systems) after which they end up 
with a ``quantum key,'' i.e., a number of quantum bits in a 
maximally entangled state. Measuring them finally 
leads to a (classical) secret key. The first phase of this protocol
is called {\em quantum (entanglement) purification}.

In order to understand what happens in  a purification protocol
and for which initial states such a protocol is at all possible, we 
recall some basic facts about quantum (information) theory. 
In contrast to a classical 
bit ({\em Cbit\/} for short) which can take either 
of the values $0$ or $1$, a quantum bit ({\em Qbit\/})
can exist in a superposition of these two extremal states
(with complex {\em probability amplitudes\/} $a$ and $b$ satisfying 
$|a|^2+|b|^2=1$):
\[
|\psi\rangle = a|0\rangle +b|1\rangle\ .
\]
When measuring this state with respect to the basis $\{|0\rangle,|1\rangle\}$,
we obtain $|0\rangle$ with probability $|a|^2$ and $|1\rangle$ otherwise.
All  (pure) states of one Qbit can  be represented 
as unit vectors in the Hilbert space ${\bf C}^2$.

A possible state of a system of two Qbits can be 
\[
|\psi\rangle= |\psi_1\rangle\otimes|\psi_2\rangle=:|\psi_1\psi_2\rangle\ ,
\]
which is simply the tensor product of the states 
$|\psi_1\rangle$ and $|\psi_2\rangle$ of the 
first and second Qbit, respectively. 
Such a state is called a {\em product state}.
However, (normalized) 
linear combinations of quantum states lead to additional states;
for instance,
\[
|\psi^{-}\rangle:=\left(|01\rangle -|10\rangle\right)/\sqrt{2}
\]
is also a  possible state of the two-Qbit system.
This state 
is called {\em singlet state\/} and 
has the property that whenever the Qbits are measured with 
respect to the same basis, the outcomes are opposite bits. 
There is no classical explanation for this behavior which
 is called 
{\em (maximal) entanglement}. We conclude that two Qbits are not the same 
as ``two times one Qbit.''


As described above, the objective of Alice and Bob  doing quantum 
purification is to generate two-Qbit systems in the state $|\psi^{-}\rangle$
(or in states very close to it) by classical communication and local 
quantum operations. The states they start with can for instance be their 
view of a pure state $|\Psi\rangle$ living in Alice's, Bob's, and a 
possible adversary Eve's (who is assumed to have total control over 
the entire environment) Hilbert spaces:
\[
|\Psi\rangle\in\HA\otimes \HB\otimes\HE\ .
\]
In analogy to Alice and Bob's marginal distribution $P_{XY}$ in the 
classical setting, one can define Alice and Bob's view of the state 
$|\Psi\rangle$, the so-called {\em trace over the environment $\HE$},
\[
\rho_{AB}:=\Tr_{\HE}(|\Psi\rangle)\ .
\]
The state $\rho_{AB}$
is generally a {\em mixed state}. In contrast to a pure state, which can be 
represented by a vector in a Hilbert space, a mixed state 
is described by a probability distribution over such a space.
A mixed state, such as $\rho_{AB}$, can be represented by a
$(\dim \HA)\cdot (\dim \HB)\times(\dim \HA)\cdot (\dim \HB)$ matrix, 
namely the weighted sum (with respect to the probability distribution)
of the projectors to the 
subspaces generated by the 
corresponding pure states. This matrix is 
called {\em density matrix}.

It is important to note that ``purification,'' which transforms the mixed 
state $\rho_{AB}$ into pure (singlet) states, actually 
means key agreement: 
 Alice and Bob's  final state is pure and hence not entangled 
with anything else, in particular not with anything under Eve's control.
The adversary is out of the picture, whatever 
operations and measurements she performs.




Let us consider some properties of mixed states. A state $\rho_{AB}$
which is {\em separable}, i.e., a mixture of product states, can be prepared
remotely by purely classical communication.
States that are not separable are called {\em entangled\/} and cannot 
be prepared this way.
It is a natural question which states $\rho_{AB}$ {\em can be purified\/} and 
which cannot. Separable states cannot be purified because of the property 
just described and because of the generalization of Shannon's theorem
mentioned at he beginning of this paper: No information-theoretic 
key agreement is possible from authentic but public (classical)
communication. On the other hand, if Alice's
and Bob's subsystems  are two-dimensional\footnote{The same is even true
if one of the spaces has dimension two and the other one has dimension
three.} (i.e., Qbits) and entangled, 
then purification is always possible~\cite{hohoho97}. 
However, the surprising fact was 
recently discovered that the same is not true for higher-dimensional 
systems: There exist entangled states which cannot be 
purified~\cite{HorodeckiPartialTransp}. 
(This follows from the fact that the eigenvalues of the 
so-called  partial transposition of  certain entangled density 
matrices $\rho_{AB}$ are  non-negative~\cite{Peres}.)
This type of entanglement is 
called {\em bound\/} (in contrast to {\em free\/} entanglement, which 
{\em can\/} be purified).
>From the perspective of classical information theory,
the interesting point is that  
bound entanglement seems to have a classical 
counterpart with unexpected properties.








\section{Linking Classical and Quantum Key Agreement}
\label{linking}

In this section we derive a close connection 
 between the possibilities offered by classical
and quantum protocols for key agreement. The intuition is as follows. 
First of all,
there is a very natural connection between quantum states $\Psi$ 
 and 
classical distributions $P_{XYZ}$ which 
can be thought of as arising from $\Psi$
by measuring in a certain basis, e.g., the standard basis\footnote{A 
priori, there is no
privileged basis. However,  physicists often write states like $\rho_{AB}$ in a
basis which seems to be more natural than others. 
We refer to this   as the standard
basis. Somewhat surprisingly, this basis is generally easy to identify,
though not precisely defined. One could  characterize the
standard basis as the basis for which as many coefficients as possible 
of $\Psi$ are
real and positive.
We usually represent quantum states with respect to the standard basis.}.
Such a measurement
leads to 
classical information with some 
probability distribution depending on the quantum state. 
In the following, we assume that Eve 
is free  
to carry out so-called {\em generalized measurements\/} (POVMs)~\cite{Peresbook}. In other words,
the set $\{|z\rangle\}$ will not be  assumed to be an orthonormal
 basis, but any set generating the Hilbert space
$\HE$ and satisfying the condition
$\sum_z|z\rangle\langle z|=\opone_{\HE}$. 
Then, if  the three parties carry out measurements in certain 
bases $\{|x\rangle\}$ and
$\{|y\rangle\}$, and in the set $\{|z\rangle\}$, respectively, 
they end up with the classical scenario 
$P_{XYZ}=|\langle x,y,z|\Psi\rangle|^2$. 


When given a state $\Psi$ between three parties 
Alice, Bob, and Eve, 
and if $\rho_{AB}$ denotes the resulting mixed state  after  Eve is traced out,
then the corresponding classical distribution $P_{XYZ}$
will have positive intrinsic information if and only if 
$\rho_{AB}$ is entangled. However, 
this correspondence clearly depends on the measurement bases used by Alice, Bob, and Eve.
If for instance $\rho_{AB}$ is entangled, but Alice and Bob do very unclever
measurements, then the intrinsic information may vanish. 
If on the other hand $\rho_{AB}$ is 
separable, Eve may do such bad measurements that the intrinsic information
becomes positive, despite the fact that $\rho_{AB}$ could have
been established by public discussion without any prior correlation 
(see Example~4). 
Consequently, the correspondence between
intrinsic information and entanglement must involve some optimization over all
possible measurements on all sides.

A similar  correspondence on the protocol level is supported 
by many examples, but not  rigorously proven:
The distribution $P_{XYZ}$ allows for classical key agreement if and 
only if quantum key agreement is possible starting from the state $\rho_{AB}$.

We show how these parallels allow for addressing problems
of purely classical information-theoretic nature 
with the methods of quantum information theory, and vice versa.



\subsection{Entanglement and Intrinsic Information}
\label{linkingone}

Let us first establish the connection between intrinsic information
and entanglement.   Theorem~\ref{theoeins} states  that if 
 $\rho_{AB}$ is separable, then Eve can ``force'' the  information
between
   Alice's and Bob's classical random variables (given Eve's classical random variable) 
to be zero (whatever strategy Alice and Bob use\footnote{The statement of 
Theorem~\ref{theoeins} also holds when Alice and Bob are allowed to do 
generalized measurements.}). 
   In particular, Eve can prevent classical key agreement. 



\begin{theo}\label{theoeins}
Let $\Psi\in\HA\otimes\HB\otimes\HE$ and $\rho_{AB}=\Tr_{\HE}(P_{\Psi})$.
If $\rho_{AB}$ is separable,
then there exists a generating set $\{|z\rangle\} $ of $\HE$  such that
for all bases $\{|x\rangle\} $ and $\{|y\rangle\} $ of $\HA$ and $\HB$,
respectively,
$I(X;Y|Z)=0$
holds for 
 $P_{XYZ}(x,y,z):=|\langle x,y,z|\Psi\rangle |^2$.
\end{theo}


\noi
\proof
If $\rho_{AB}$ is separable, then there exist vectors $|\alpha_z\rangle$ and
$|\beta_z\rangle$
such that
$
\rho_{AB}=\sum_{z=1}^{n_z} p_z
P_{\alpha_z}\otimes P_{\beta_z}
$,
where $P_{\alpha_z}$ denotes the one-dimensional projector onto the subspace spanned
by $|\alpha_z\rangle$.

Let us first assume that $n_z\le \dim\HE$.
Then there exists a basis $\{|z\rangle\}$ of $\HE$ such that 
$
\Psi=\sum_z \sqrt{p_z}\, |\alpha_z,\beta_z,z\rangle 
$
holds~\cite{qp9609013},\, 
\cite{gisin89},\, \cite{hjw93}.

If $n_z> \dim\HE$, then Eve can add an auxiliary system $\H_{aux}$
to hers (usually called an {\em ancilla\/}) and we have
$
\Psi\otimes|\gamma_0\rangle=\sum_z \sqrt{p_z}\,
|\alpha_z,\beta_z,\gamma_z\rangle
$,
where $|\gamma_0\rangle\in\H_{aux}$ is the state of Eve's auxiliary
system, and 
$\{|\gamma_z\rangle\}$ is a basis of $\HE\otimes\H_{aux}$.
We define the (not necessarily orthonormalized) vectors $|z\rangle$ by 
$|z,\gamma_0\rangle=\opone_{\HE}\otimes P_{\gamma_0}|\gamma_z\rangle$.
These vectors determine a generalized measurement with positive operators
$O_z=|z\rangle\langle z|$.
Since $\sum_zO_z\otimes
P_{\gamma_0}=\sum_z|z,\gamma_0\rangle\langle z,\gamma_0|
=\sum_z\opone_{\HE}\otimes
P_{\gamma_0}|\gamma_z\rangle\langle\gamma_z| \opone_{\HE}\otimes
P_{\gamma_0}
=\opone_{\HE}\otimes P_{\gamma_0}$, the $O_z$ satisfy
$\sum_zO_z=\opone_{\HE}$, as 
they should in order to define a generalized measurement \cite{Peresbook}.
Note that the first case ($n_z\le \dim\HE$) is a special case of the
second one, with $|\gamma_z\rangle=|z,\gamma_0\rangle$.
If Eve now performs the measurement, then we have $P_{XYZ}(x,y,z)=|\langle
x,y,z|\Psi\rangle |^2
=|\langle x,y,\gamma_z|\Psi,\gamma_0\rangle |^2$, and
\[
P_{XY|Z}(x,y,z)=|\langle x,y|\alpha_z,\beta_z\rangle |^2
=|\langle x|\alpha_z\rangle |^2\, |\langle y|\beta_z\rangle
|^2=P_{X|Z}(x,z)P_{Y|Z}
(y,z)
\]
holds for all $|z\rangle$ and for all
$|x,y\rangle \in\HA\otimes\HB$. Consequently, $I(X;Y|Z)=0$.
\peon 
\ 

Theorem~\ref{theozwei} states that 
 if $\rho_{AB}$ is  entangled,
   then Eve {\em cannot\/} force the intrinsic information to be zero: Whatever
   she does (i.e., whatever generalized measurements she carries out), there 
   is something Alice and Bob can do such that the intrinsic information
   is positive. Note that this does {\em not}, a priori, imply that secret-key
   agreement is possible in every case. Indeed, we will provide 
   evidence for the fact that this implication does generally {\em not\/}
   hold.    

\begin{theo}\label{theozwei}
Let $\Psi\in\HA\otimes\HB\otimes\HE$ and $\rho_{AB}=\Tr_{\HE}(P_{\Psi})$.
If $\rho_{AB}$ is entangled,
then for all generating sets $\{|z\rangle\} $ of $\HE$, 
there are bases $\{|x\rangle\} $ and $\{|y\rangle\} $ of $\HA$ and $\HB$, 
respectively,
such that
$I(X;Y\down Z)>0$
holds for
 $P_{XYZ}(x,y,z):=|\langle x,y,z|\Psi\rangle |^2$.
\end{theo}

\noi
\proof
We prove this by contradiction.
Assume 
that there exists a generating set $\{|z\rangle\}$ of $\HE$ such that for 
all bases $\{|x\rangle\}$ of $\HA$ and $\{|y\rangle\}$ of $\HB$,
we have $I(X;Y\down Z)=0$  for the resulting distribution. 
For such a distribution,  there exists a channel, characterized by
$P_{\OZ|Z}$,
such that $I(X;Y|\OZ)=0$ holds, i.e., 
\begin{equation}
P_{XY|\OZ}(x,y,\overline{z})=P_{X|\OZ}(x,\overline{z})P_{Y|\OZ}(y,\overline{z})\ .
\label{pbar}
\end{equation}
Let 
$
\rho_{\overline{z}}
:=(1/p_{\overline{z}})\sum_z p_z P_{\OZ|Z}(\overline{z},z)P_{\psi_z}
$,
$p_z=P_Z(z)$,
and
$
p_{\overline{z}}=\sum_z P_{\OZ|Z}(\overline{z},z)p_z,
$
where $\psi_z$ is the state of Alice's and Bob's system conditioned on Eve's
result $z$:
$\Psi\otimes|\gamma_0\rangle=\sum_z\psi_z\otimes|\gamma_z\rangle$
(see the proof of Theorem~1).

 From (\ref{pbar}) we can conclude
$
\Tr(P_x\otimes P_y\rho_{\overline{z}})=\Tr(P_x\otimes \opone\rho_{\bar
z})\, \Tr(\opone\otimes P_y\rho_{\overline{z}})
$
for all one-dimensional
 projectors $P_x$ and $P_y$ acting in $\HA$ and $\HB$,
respectively.
Consequently, the states $\rho_{\overline{z}}$ are products, i.e., 
$
\rho_{\overline{z}}=\rho_{\alpha_{\overline{z}}}\otimes\rho_{\beta_{\overline{z}}},
$
and $\rho_{AB}=\sum_\z p_\z \rho_\z$ is separable.
\peon \ 


Theorem~\ref{theozwei} can be formulated in a more 
positive way. Let us first 
introduce the concept of a set of bases 
$(\{|x\rangle\}_j,\{|y\rangle\}_j)$,
where the $j$
   label the different bases, as they are used in the 4-state (2 bases) and
   the 6-state (3 bases) protocols~\cite{BB84},\, \cite{Dagmar6state},\, \cite{bg}. 
Then if $\rho_{AB}$ is entangled there exists a set
$(\{|x\rangle\}_j,\{|y\rangle\}_j)_{j=1,\ldots,N}$
 of $N$
bases such that for all generalized measurements $\{|z\rangle\}$,
$
I(X;Y\down [Z,j])>0
$
holds.
The idea is that Alice and Bob randomly choose a basis and, after the
transmission, publicly restrict to the (possibly few) cases where they
happen to have
chosen the same basis. Hence Eve knows $j$, and one has
\[
I(X;Y\down [Z,j])=\frac{1}{N}\sum_{j=1}^N I(X^j;Y^j\down Z)\ .
\]
If the set of bases is large enough, then for all $\{|z\rangle\}$
 there is a basis with
positive intrinsic information, 
hence the mean is also positive. 
Clearly, this result is 
stronger if the set of bases is small. 
Nothing is proven about 
the achievable size of such sets of bases, but it is conceivable that
$\max\{\dim\HA,\dim\HB\}$
 bases
are always sufficient.

It is important to note in this context that when the measurements 
are actually carried out by the parties, then Alice and Bob can 
obtain positive intrinsic information only  if Eve cannot 
choose her measurement basis adaptively 
(i.e., {\em after\/} learning what bases
Alice and Bob have used). 


\begin{cor}\label{coreinszwei}
Let $\Psi\in\HA\otimes\HB\otimes\HE$ and $\rho_{AB}=\Tr_{\HE}(P_{\Psi})$.
Then the following statements
are equivalent:
\\ \ \\
\quad
{\it (i)}
$\rho_{AB}$ is entangled,
\\ \ \\
\quad
{\it (ii)}
for all generating sets $\{|z\rangle\}$ of $\HE$, there exist bases
$\{|x\rangle\}$ of $\HA$ and 
$\{|y\rangle\}$ of  $\HB$ such that the 
distribution  $P_{XYZ}(x,y,z):=|\langle x,y,z|\Psi\rangle|^2$ satisfies 
$\ida>0$,
\\ \ \\
\quad
{\it (iii)}
for all generating sets  $\{|z\rangle\}$ of $\HE$, there exist bases
$\{|x\rangle\}$ of $\HA$
and $\{|y\rangle\}$ of  $\HB$ such that the 
distribution  $P_{XYZ}(x,y,z):=|\langle x,y,z|\Psi\rangle|^2$ satisfies 
$I(X;Y|Z)>0$.
\end{cor}


A first consequence of the fact that the statement of 
Corollary~\ref{coreinszwei} often holds
with respect to the standard bases (see below) is that it yields, at least 
in the binary case, a criterion for $\ida>0$ that is efficiently verifiable 
since it is based on the positivity of the eigenvalues of 
the partial transpose of the density matrix, i.e., of 
a $4\times 4$ matrix. 
Previously, the quantity $\ida$ has been considered hard to handle.





\subsection{Examples I}
\label{exampleseins}

The following examples illustrate the correspondence established in 
Section~\ref{linkingone}. 
They show in particular that very
often (Examples 1, 2, and
3), but not always (Example 4), the direct connection between 
entanglement and positive intrinsic information holds with respect 
to the standard bases (i.e., the bases physicists use by commodity
and intuition).
Example~1 was already analyzed in~\cite{giswol99}. 
The  examples of this section will be  discussed further   in 
Section~\ref{exampleszwei} under the aspect of the existence of key-agreement 
protocols in the classical and quantum regimes.
\ \\



\noi
{\it Example 1.}
Let us consider the  so-called 4-state  protocol
of~\cite{BB84}.
The analysis of the  6-state protocol~\cite{bg} is analogous and leads to similar
results. 
We compare  the possibility of 
quantum and classical key agreement given the 
quantum state and the corresponding classical distribution, respectively, arising 
from this protocol. The conclusion 
is, under the assumption of incoherent eavesdropping, that key 
agreement in one setting is possible if and only if this is true 
also for the other. 

After carrying out the 4-state protocol, and under the assumption of 
optimal  eavesdropping (in terms of Shannon information), the resulting 
quantum state is~\cite{FGGNP}
{\small
\[
\Psi=\sqrt{F/2}\, |0,0\rangle\otimes \xi_{00}+\sqrt{D/2}\, |0,1\rangle\otimes  \xi_{01}+\sqrt{D/2}\, |1,0\rangle\otimes  \xi_{10}+
\sqrt{F/2}\, |1,1\rangle\otimes  \xi_{11}\ ,
\]
}where $D$ (the {\em disturbance\/}) is the probability that $X\ne Y$ holds if
 $X$ and $Y$ are the classical random variables of Alice and Bob, respectively,
where $F=1-D$ (the {\em fidelity\/}),
and where the $\xi_{ij}$ satisfy $\langle \xi_{00}|\xi_{11}\rangle=
\langle \xi_{01}|\xi_{10}\rangle=1-2D$ and $\langle \xi_{ii}|\xi_{ij}\rangle=0$
for all $i\ne j$. Then the state $\rho_{AB}$ is (in the  basis
$\{\ket{00}$,\, $\ket{01}$,\, $\ket{10}$,\, $\ket{11}\}$)
{\small
\[
\rho_{AB}=\frac{1}{2}\left(
\begin{array}{cccc}
\D & 0 & 0 & -\D(1-2D) \\
0 & 1-D & -(1-D)(1-2D) & 0\\
0 & -(1-D)(1-2D) & 1-D & 0\\
-\D(1-2D) & 0 & 0 & \D
\end{array}
\right)
\]
}and its partial transpose
{\small
\[
\rho_{AB}^t=
\frac{1}{2}\left(
\begin{array}{cccc}
\D & 0 & 0 & -(1-D)(1-2D) \\
0 & 1-D & -\D(1-2D) & 0 \\
0 & -\D(1-2D) & 1-D & 0 \\
-(1-D)(1-2D)  & 0 & 0 & \D
\end{array}
\right)
\]
}has the eigenvalues $(1/2)(D\pm(1-D)(1-2D))$ and $(1/2)((1-D)\pm D(1-2D))$, which are 
all non-negative (i.e., $\rho_{AB}$ is separable) if
\begin{equation}\label{coeins}
D\geq 1-\frac{1}{\sqrt{2}}\ .
\end{equation}



>From the classical viewpoint, the corresponding distributions (arising
 from measuring the above quantum system in the standard bases) 
are as follows. 
First, $X$ and $Y$ are both symmetric bits with $\prob[X\ne Y]=\cd$. 
Eve's random variable $Z=[Z_1,Z_2]$ is composed of 2 bits $Z_1$ and $Z_2$,
where $Z_1=X\oplus Y$, i.e., $Z_1$ tells Eve whether Bob received the qubit
disturbed ($Z_1=1$) 
or not ($Z_1=0$) (this is a consequence of the fact that the $\xi_{ii}$ and
$\xi_{ij}$ ($i\ne j$) states
 generate orthogonal sub-spaces), and where
the probability that Eve's second bit indicates the correct value
of Bob's bit 
is
Prob$[Z_2=Y]=\delta=(1+\sqrt{1-\langle \xi_{00}|\xi_{11}\rangle^2})/2=1/2+\sqrt{D(1-D)}$. 
We now prove that for this distribution, 
the intrinsic information is zero 
if and only if 
\begin{equation}\label{bsbed}
\frac{\cd}{1-\cd}\geq 2\sqrt{(1-\de)\de}=1-2D
\end{equation}
holds. 
We show that if the condition (\ref{bsbed})
 is 
satisfied, then $\ida=0$ holds. 
The inverse implication follows from the existence of a key-agreement 
protocol in all other cases (see Example~1 (cont'd) in Section~\ref{exampleszwei}).
If  (\ref{bsbed})  holds, we can construct a random variable $\OZ$,
that is generated by sending $Z$ over a channel characterized by
$P_{\OZ|Z}$,  for which  $I(X;Y|\OZ)=0$ holds. 
We can restrict ourselves to the case of equality  in (\ref{bsbed})
because Eve can always increase $\de$ by adding noise. 

Consider now the channel characterized by the following conditional 
distribution
$P_{\OZ|Z}$ (where $\overline{{\cal Z}}=\{u,v\}$):
\begin{eqnarray*}
P_{\OZ|Z}(u,[0,0])&  =&  P_{\OZ|Z}(v,[0,1]) = 1\ ,\\
 P_{\OZ|Z}(l,[1,0]) &   = &   P_{\OZ|Z}(l,[1,1])  =  1/2
\end{eqnarray*}
for $l\in\{u,v\}$.
We show $I(X;Y|\OZ)={\rm E}_{\OZ}\, [I(X;Y|\OZ=\oz)]=0$, i.e., that 
$I(X;Y|\OZ=u)=0$ and $I(X;Y|\OZ=v)=0$ hold. By symmetry it is sufficient 
to show the first equality. For  $a_{ij}:=P_{XY\OZ}(i,j,u)$, we get
\begin{eqnarray*}
a_{00} &   = & (1-D)(1-\de)/2\, ,\\
a_{11}&  = & (1-D)\de/2\, ,\\ 
a_{01}&  = &  a_{10}\ =\ 
(D(1-\de)/2+D\de/2)/2=D/4\, .
\end{eqnarray*}
>From  equality  in (\ref{bsbed})  we conclude 
$
a_{00}a_{11}=a_{01}a_{10}
$,
which is equivalent to the fact that $X$ and $Y$ are independent, given
$\OZ=u$.

Finally, note that the conditions (\ref{coeins}) and (\ref{bsbed}) are 
equivalent for $D\in [0,1/2]$. This shows that the bounds of tolerable noise
are indeed  the same for the quantum and classical scenarios.
\exend
\\ \

\noi
{\it Example 2.}
We consider the bound entangled state presented in~\cite{HorodeckiPartialTransp}. This
example received quite a lot of attention by the quantum-information community
because it was the first known example of bound entanglement (i.e., entanglement without
the possibility of
quantum key agreement). We show that 
its classical counterpart seems to have similarly surprising properties.
 Let $0<a<1$ and
\begin{eqnarray*}
\Psi & = & \sqrt{\frac{3a}{8a+1}}\, \psi\otimes|0\rangle  +
\sqrt{\frac{1}{8a+1}}\, \phi_a\otimes|1\rangle  + \\
&& \qquad
+\sqrt{\frac{a}{8a+1}}\, (|122\rangle +|133\rangle +|214\rangle +|235\rangle +|326\rangle )\ ,
\end{eqnarray*}
where $\psi=(|11\rangle +|22\rangle +|33\rangle )/\sqrt{3}$ and 
$
\phi_a=\sqrt{(1+a)/2}\, |31\rangle +\sqrt{(1-a)/2}\, |33\rangle.
$
It has been shown in~\cite{HorodeckiPartialTransp} that the resulting state 
$\rho_{AB}$
is entangled.

The
corresponding classical
distribution is 
 as follows. The ranges are
$
\CX=\CY=\{1,2,3\}$ and $\CZ=\{0,1,2,3,4,5,6\}
$.
We write
$
(ijk)=P_{XYZ}(i,j,k)
$.
Then we have
$
(110)=(220)=(330)=(122)=(133)=(214)=(235)=(326)  =  2a/(16a+2)$,
$(311)  =  (1+a)/(16a+2)$, and 
$(331)  =  (1-a)/(16a+2)$.
We study the special case $a=1/2$. Consider the following representation
of the resulting distribution (to be normalized).
For instance,
the entry ``$(0)\ 1\ ,\ (1)\ 1/2$'' for $X=Y=3$  means 
$P_{XYZ}(3,3,0)=1/10$ (normalized),
$P_{XYZ}(3,3,1)=1/20$,
and  $P_{XYZ}(3,3,z)=0$ for all $z\not\in\{0,1\}$.
\\
\begin{center}
\begin{tabular}{|c||c|c|c|}
\hline
\ \ X & 1 & 2 & 3\\
Y (Z) &&&\\
\hline\hline
1 & (0)\ 1 & (4)\ 1 & (1)\ 3/2\\
\hline
2 & (2)\ 1 & (0)\ 1 & (6)\ 1\\
\hline
3 & (3)\ 1 & (5)\ 1 & (0)\ 1\\
&&& (1)\ 1/2\\
\hline
\end{tabular}
\end{center}



As we would expect, the  intrinsic information is positive 
in this scenario. This can be seen by contradiction as follows. 
Assume $\ida=0$. Hence there exists a discrete channel, characterized
by the conditional distribution $P_{\OZ|Z}$, such that $I(X;Y|\OZ)=0$ holds.
Let $\overline{\CZ}\subseteq {\bf N}$
be the range of $\overline{Z}$, and let
$
P_{\OZ|Z}(i,0)  =:  a_i$,
$P_{\OZ|Z}(i,1)  =:  x_i$,
$P_{\OZ|Z}(i,6)  =:  s_i$.
Then we must have 
$a_i,x_i,s_i\in [0,1]$ and 
$\sum_i{a_i}=\sum_i{x_i}=\sum_i{s_i}=1$.
Using  $I(X;Y|\OZ)=0$, we obtain the following distributions 
$P_{XY|\OZ=i}$ (to be normalized):
\\
\begin{center}
\begin{tabular}{|c||c|c|c|}
\hline
\ \ X & 1 & 2 & 3\\
Y  &&&\\
\hline\hline
1 & $a_i$ & $\frac{3a_ix_i}{2s_i}$ & $\frac{3x_i}{2}$\\
\hline
2 & $\frac{2a_is_i}{3x_i}$ & $a_i$ & $s_i$\\
\hline
3 & $\frac{2a_i(a_i+x_i/2)}{3x_i}$ & $\frac{a_i(a_i+x_i/2)}{s_i}$ & 
$a_i+\frac{x_i}{2}$\\
\hline
\end{tabular}
\end{center}
By comparing the $(2,3)$-entries of the two tables above, we obtain
\begin{equation}\label{drei}
1\geq\sum_i{\frac{a_i(a_i+x_i/2)}{s_i}}\ .
\end{equation}

We  prove that  (\ref{drei}) 
implies $s_i\equiv a_i$ (i.e., $s_i =a_i$ for all $i$) and $x_i\equiv 0$. 
Clearly, this 
does not lead to a solution and is hence
a 
contradiction. For instance,
$
P_{XY|\OZ=i}(1,2)=2a_is_i/3x_i
$
is not even defined in this case if $a_i>0$.

It remains to show that  (\ref{drei})
implies $a_i\equiv s_i$ and $x_i\equiv 0$. We show that 
whenever $\sum_i{a_i}=\sum_i{s_i}=1$ and $a_i\not\equiv s_i$, then
$
\sum_i{a_i^2/s_i}>1\ .
$
First, note that $\sum_i{a_i^2/s_i}=\sum_i{a_i}=1$ for $a_i\equiv s_i$.
Let now $s_{i_1}\leq a_{i_1}$ and $s_{i_2}\geq a_{i_2}$. We show that 
$
a_{i_1}^2/s_{i_1}+a_{i_2}^2/s_{i_2}<a_{i_1}^2/(s_{i_1}-\ep)+a_{i_2}^2/(s_{i_2}+\ep)
$
holds for every $\ep>0$,
which obviously  implies  the above statement. It is  straightforward 
to see that this
 is equivalent to
$
a_{i_1}^2s_{i_2}(s_{i_2}+\ep)>a_{i_2}^2s_{i_1}(s_{i_1}-\ep),
$
and holds because of
$
a_{i_1}^2s_{i_2}(s_{i_2}+\ep)>a_{i_1}^2a_{i_2}^2$ and 
$
a_{i_2}^2s_{i_1}(s_{i_1}-\ep)<a_{i_1}^2a_{i_2}^2
$.
This concludes the proof of  $\ida>0$.
\exend
\ \\

As mentioned,
the interesting point about Example~2 is that the quantum state is
bound entangled, and that also classical key agreement seems 
impossible despite the fact that $\ida>0$ holds. This is a  contradiction 
 to a conjecture stated in~\cite{ittrans}. 
The classical translation of the bound entangled state leads to a classical 
distribution with very strange properties as well! (See 
Example~2 (cont'd) in Section~\ref{exampleszwei}).

In
Example~3, another  bound entangled state (first proposed in~\cite{dreih}) is discussed. 
The example  is particularly 
nice because, depending on the choice of a parameter $\al$, the quantum 
state can be made separable, bound entangled, and free entangled.
\\ 




\noi
{\it Example 3.}
We consider the following  distribution (to be normalized).
Let $0\leq \al \leq 3$.
\\
\begin{center}
\begin{tabular}{|c||c|c|c|}
\hline
\ \ X & 1 & 2 & 3\\
Y (Z) &&&\\
\hline\hline
1 & (0)\ 2 & (4)\ $5-\al$ & (3)\ $\al$\\
\hline
2 & (1)\ $\al$ & (0)\ 2 & (5)\ $5-\al$\\
\hline
3 & (6)\ $5-\al$ & (2)\ $\al$ & (0)\ 2\\
\hline
\end{tabular}
\end{center}
This distribution arises when measuring the following quantum state.
Let 
$
\psi:=(1/\sqrt{3})\, (|11\rangle+|22\rangle+|33\rangle).
$
Then 
\begin{eqnarray*}
\Psi_{\al} & = & \sqrt{\frac{2}{7}}\, \psi\otimes|0\rangle + 
\sqrt{\frac{a}{21}}\, (|12\rangle\otimes|1\rangle+|23\rangle\otimes|2\rangle
+|31\rangle\otimes|3\rangle)\\
&& 
+ 
\sqrt{\frac{5-\al}{21}}\, (|21\rangle\otimes|4\rangle 
+|32\rangle\otimes|5\rangle
+|13\rangle\otimes|6\rangle),\qquad\quad\mbox{and}\\
\rho_{AB} & = & \frac{2}{7}\, P_{\psi} + \frac{\al}{21}\, 
(P_{12}+P_{23}+P_{31}) + \frac{5-\al}{21}\, (P_{21}+P_{32}+P_{13})
\end{eqnarray*}
is separable if and only if $\al\in[2,3]$, bound entangled for $\al\in [1,2)$, 
and 
free entangled if $\al\in[0,1)$~\cite{dreih} (see Figure~\ref{qcw}).

Let us consider the quantity $\ida$. First of all, it is clear that 
$\ida=0$ holds for $\al\in[2,3]$. The reason is that 
$
\al\geq 2$ and $5-\al\geq 2
$
together imply that Eve can ``mix'' her symbol $Z=0$ with the remaining 
symbols  in such a way that when given that $\OZ$ takes the ``mixed value,'' then
$XY$ is uniformly distributed; in particular, $X$ and $Y$ are 
independent. Moreover, it can be shown in analogy to 
Example 2 that $\ida>0$ holds for  $\al<2$.
\exend
\\ \



Examples~1, 2, and~3 suggest that the correspondence between separability
and entanglement on one side and vanishing and non-vanishing intrinsic 
information on the other always holds with respect to the standard bases
or even arbitrary bases.
This is however not true in general: 
Alice and Bob as well as Eve can perform bad measurements 
and give away an initial advantage. 
The following is   a simple example where 
measuring in the standard basis is a bad choice for Eve.
\\ \

\noi
{\it Example 4.}
Let us consider the quantum states
\begin{eqnarray*}
\Psi &  = &  \frac{1}{\sqrt{5}}\, (|00+01+10\rangle \otimes|0\rangle  + |00+11\rangle \otimes|1\rangle )\ ,\\
\rho_{AB}&  =&  \frac{3}{5}\,  P_{|00+01+10\rangle } + \frac{2}{5}\,  P_{|00+11\rangle }\ .
\end{eqnarray*}
If Alice, Bob, and Eve measure in the standard bases, we get the classical distribution 
 (to be normalized)
\begin{center}
\begin{tabular}{|c||c|c|}
\hline
\ \ X & 0 & 1\\
Y (Z) &&\\
\hline\hline
0 & (0)\ 1 & (0)\ 1 \\
 &  (1)\ 1 & (1)\ 0 \\
\hline
1 & (0)\ 1 & (0)\ 0 \\
  & (1)\ 0 & (1)\ 1\\
\hline
\end{tabular}
\end{center}
For this distribution, $\ida>0$ holds. Indeed, 
even $S(X;Y||Z)>0$ holds.
This is not surprising since both $X$ and $Y$ 
are binary, and since the described parallels suggest that in this 
case, positive intrinsic information implies that a secret-key 
agreement protocol exists. 

The proof of $S(X;Y||Z)>0$ in this situation is analogous to the proof of this 
fact in Example~3. The protocol consists of Alice and Bob independently 
making their bits symmetric. Then the repeat-code protocol can be 
applied. 

However, the 
partial-transpose condition shows that $\rho_{AB}$ is separable. This 
means that measuring in the standard basis is bad for Eve. Indeed,
let us rewrite $\Psi$ and $\rho_{AB}$ as
\begin{eqnarray*}
\Psi &  = &   \sqrt{\Lambda}\, |m,m\rangle \otimes|\tilde 0\rangle + 
     \sqrt{1-\Lambda}\, |-m,-m\rangle \otimes|\tilde 1\rangle\ ,\\
\rho_{AB} &   =&   \frac{5+\sqrt{5}}{10}\,  P_{|m,m\rangle } + 
         \frac{5-\sqrt{5}}{10}\,  P_{|-m,-m\rangle }\ ,
\end{eqnarray*}
where
$\Lambda  =  (5+\sqrt{5})/10$, 
$|m,m\rangle  =  |m\rangle \otimes|m\rangle$,
$|\pm m\rangle  =  \sqrt{(1\pm\eta)/2}\, |0\rangle  \pm \sqrt{(1\mp\eta)/2}\, |1\rangle$, and 
$\eta  =  1/\sqrt{5}$.

In this representation, $\rho_{AB}$ is obviously separable. It also means that 
 Eve's optimal measurement basis is
\[
|\tilde 0\rangle =\sqrt{\Lambda}\, |0\rangle  - \frac{1}{\sqrt{5\Lambda}}\, |1\rangle \ ,\ \ 
|\tilde 1\rangle =-\sqrt{1-\Lambda}\, |0\rangle  - \frac{1}{\sqrt{5(1-\Lambda)}}\, |1\rangle\ . 
\]
Then, $\ida=0$ holds for the resulting classical distribution.
\exend
\ \\



  \subsection{A Classical Measure for Quantum Entanglement}
\label{linkingmeasure}

It is a
challenging problem
of theoretical quantum physics to find good measures for entanglement~\cite{qp9610044}.
Corollary~\ref{coreinszwei} above
   suggests the following measure,  which is based on
    classical information theory. 
\begin{defi}
{\rm
Let for a quantum state $\rho_{AB}$
\[
\mu(\rho_{AB}):= \min_{\{|z\rangle\}}\, (\max_{\{|x\rangle\},\{|y\rangle\}}\, (\ida))\ ,
\]
where the minimum is taken over all $\Psi=\sum_z\sqrt{p_z}\psi_z\otimes|z\rangle $
such that
$\rho_{AB}=\Tr_{\HE}(P_{\Psi})$ holds
and over all generating sets 
 $\{|z\rangle\}$ of $\HE$, the maximum is over all bases $\{|x\rangle\}$
of $\HA$ and $\{|y\rangle\}$
of $\HB$, and where
$P_{XYZ}(x,y,z):=|\langle x,y,z|\Psi\rangle |^2$.
}
\defend
\end{defi}

The function $\mu$ has all the properties required from such a measure.
If $\rho_{AB}$ is pure, i.e., $\rho_{AB}=|\psi_{AB}\rangle\langle\psi_{AB}|$, 
then we have  in the Schmidt
basis (see for example~\cite{Peresbook}) 
$\psi_{AB}=\sum_j c_j |x_j,y_j\rangle $, and
$\mu(\rho_{AB})=-\Tr(\rho_A\log\rho_A)$ (where 
$\rho_A=\Tr_B(\rho_{AB})$)
 as it should~\cite{qp9610044}.
It is obvious that $\mu$ is convex, i.e., 
$\mu(\lambda\rho_1+(1-\lambda)\rho_2)\le\lambda\mu(\rho_1)+(1-\lambda)\mu(
\rho_2)$.
\\ \

\noi
{\it Example~5.} 
This example is   based on Werner's states.
Let
$
\Psi=\sqrt{\lambda}\, \psi^{(-)}\otimes|0\rangle  +
\sqrt{(1-\lambda)/4}\, |001+012+103+114\rangle
$,
where $\psi^{(-)}=|10-01\rangle /\sqrt{2}$, and $\rho_{AB}=\lambda P_{\psi^{(-)}} +
((1-\lambda)/4)\opone$.
It is well-known that $\rho_{AB}$ is separable if and only if 
 $\lambda\le 1/3$.
Then the classical distribution is $P(010)=P(100)=\lambda/2$ and 
$P(001)=P(012)=P(103)=P(114)=(1-\lambda)/4$.




If $\lambda\le1/3$, then consider the channel
$P_{\OZ|Z}( 0,0)=
P_{\OZ|Z}(  2,2)=
P_{\OZ|Z}(  3,3)=1\, ,\ 
P_{\OZ|Z}(  0,1)=
P_{\OZ|Z}(  0,4)=\xi\, ,\ 
P_{\OZ|Z}(  1,1)=
P_{\OZ|Z}(  4,4)=1-\xi\, ,
$ 
where $\xi=2\lambda/(1-\lambda)\le 1$. 
Then  $\mu(\rho_{AB})=\ida=I(X;Y|\OZ)=0$ holds, as it should.

If $\lambda>1/3$, then consider the (obviously optimal) channel
$P_{\OZ|Z}(  0,0)=
P_{\OZ|Z}(  2,2)=
P_{\OZ|Z}(  3,3)=
P_{\OZ|Z}(  0,1)=
P_{\OZ|Z}(  0,4)=1
$.
Then
\begin{eqnarray*}
\mu(\rho_{AB}) & = & \ida=I(X;Y|\OZ)=P_{\OZ}(0)\cdot I(X;Y|\OZ=0)\\
& = & \frac{1+\lambda}{2}\cdot 
(1-q\log_2 q-(1-q)\log_2(1-q))\ ,
\end{eqnarray*}
where $q=2\lambda/(1+\lambda)$.
\exend


  \subsection{Classical Protocols and Quantum Purification}\label{prot}

It is a natural question  whether the analogy between entanglement and
intrinsic information (see Section~\ref{linkingone}) carries over to the 
protocol level. The examples given in Section~\ref{exampleszwei} support this 
belief.
A quite interesting and surprising consequence 
 would be that there exists a classical counterpart to bound 
entanglement, namely intrinsic information that cannot be distilled 
into a secret key by any classical protocol, if $|\cx|+|\cy|> 5$,
where $\cx$ and $\cy$ are the ranges of $X$ and $Y$, respectively.
In other words, the conjecture in~\cite{ittrans}  that such
information can always be distilled would be {\em proved\/} for 
$|\cx|+|\cy|\leq 5$, but {\em disproved\/} otherwise. 

\begin{con}\label{conprot}
Let $\Psi\in\HA\otimes\HB\otimes\HE$ and $\rho_{AB}=\Tr_{\HE}(P_{\Psi})$.
Assume that  for all generating sets  $\{|z\rangle\}$ of $\HE$ there 
are bases $\{|x\rangle\}$ and $\{|y\rangle\}$ of $\HA$ and $\HB$, respectively, 
such that 
$
S(X;Y||Z)>0
$
holds for the distribution $P_{XYZ}(x,y,z):=|\langle x,y,z|\Psi\rangle|^2$.
Then quantum purification is possible with the state $\rho_{AB}$, 
i.e., $\rho_{AB}$ is free entangled.
\end{con}

\begin{con}
Let $\Psi\in\HA\otimes\HB\otimes\HE$ and $\rho_{AB}=\Tr_{\HE}(P_{\Psi})$.
Assume that  there exists a  generating set $\{|z\rangle\}$ of $\HE$
such that for all 
 bases $\{|x\rangle\}$ and $\{|y\rangle\}$ of $\HA$ and $\HB$, respectively, 
$
S(X;Y||Z)=0
$
holds for the distribution $P_{XYZ}(x,y,z):=|\langle x,y,z|\Psi\rangle|^2$.
Then quantum purification is impossible with the state $\rho_{AB}$, 
i.e., $\rho_{AB}$ is bound entangled or  separable.
\end{con}






  \subsection{Examples II}
\label{exampleszwei}

The following examples 
support  Conjectures~1 and~2 and illustrate  their
 consequences.
We consider mainly the same distributions as in Section~\ref{exampleseins},
but this time under the aspect of the existence of classical and quantum 
key-agreement protocols. 
\\ \






\noi
{\it Example 1 (cont'd).}
We have shown in Section~\ref{exampleseins} that the 
resulting quantum state is entangled if and only if the intrinsic information
of the corresponding classical situation (with respect to the standard 
bases) is non-zero. Such a correspondence also holds on the 
protocol level. First of all, it is clear  for the quantum state that 
QPA is possible whenever the state is entangled because  both $\HA$ and $\HB$
have dimension two.
On the other hand,  the same is also true for 
the corresponding classical situation, i.e.,  secret-key agreement is possible 
whenever
$
\cd/(1-\cd)<2\sqrt{(1-\de)\de}
$
holds, i.e., if the intrinsic information is positive.
The necessary protocol includes an interactive phase, called
{\em advantage distillation}, based on a repeat code or 
on parity checks (see~\cite{ka}
or~\cite{diss}). 
\exend
\ \\ 


\noi
{\it Example 2 (cont'd).}
 The quantum state $\rho_{AB}$ in this example  is bound
entangled, meaning that the entanglement cannot be used for QPA. Interestingly,
but not surprisingly given the discussion above, the corresponding classical 
distribution has the property that $\ida>0$, but nevertheless, all the known 
classical advantage-distillation protocols~\cite{ka},\,  \cite{ittrans}
 fail for this distribution! It seems
that $S(X;Y||Z)=0$ holds (although it is not clear  how this fact could 
be rigorously proven).
\exend
\\ \

\noi
{\it Example 3 (cont'd).}
 We have seen already
that for $2\leq \al\leq 3$, the quantum state is separable and the 
corresponding classical distribution (with respect to the standard bases)
has vanishing intrinsic information. Moreover, it has been shown that for the
quantum situation, $1\leq \al<2$ corresponds to bound entanglement, 
whereas for $\al<1$, QPA is possible and allows
for generating  a secret key~\cite{dreih}. We describe a classical 
protocol here which suggests
that the situation for the classical translation of the scenario
is totally analogous: The  
protocol allows classical key agreement exactly for $\al<1$. 
However, this does not imply (although it appears very plausible) that 
no classical protocol exists at all for the case $\al\geq 1$.

Let $\al<1$. We consider the following protocol for classical key agreement.
First of all, Alice and Bob both restrict their ranges to $\{1,3\}$ (i.e.,
publicly reject a realization unless $X\in\{1,3\}$ and  $Y\in\{1,3\}$). 
We will later call this a {\em binarization\/} of the corresponding 
random variables and show that this concept  is of great importance 
in the context of possibility and impossibility of secret-key 
agreement, both classical and quantum.

The
resulting distribution is as follows (to be normalized):
\begin{center}
\begin{tabular}{|c||c|c|}
\hline
\ \ X & 1 & 3\\
Y (Z) &&\\
\hline\hline
1 & (0)\ 2 & (4)\ $\al$ \\
\hline
3 & (2)\ $5-\al$& (0)\ 2 \\
\hline
\end{tabular}
\end{center}
Then, Alice and Bob both send their bits locally over channels $P_{\OX|X}$
and $P_{\OY|Y}$, respectively, such that the resulting bits $\OX$ and 
$\OY$ are symmetric. The channel $P_{\OX|X}$ [$P_{\OY|Y}$] sends $X=0$
[$Y=1$] to $\OX=1$ [$\OY=0$] with probability $(5-2\al)/(14-2\al)$, and
leaves $X$ [$Y$] unchanged otherwise. The  distribution $P_{\OX\OY Z}$ is then
{\small
\begin{center}
\begin{tabular}{|c||c|c|}
\hline
\ \ $\OX$ & 1 & 3\\
$\OY$ (Z) &&\\
\hline\hline
 & (0)\ $2\cdot\frac{9}{14-2\al}$ & (1)\ $\al$\\
1 & (2)\ $(5-\al)\cdot\frac{9}{14-2\al}\cdot\frac{5-2\al}{14-2\al}$ 
& (2)\ $(5-\al)\left(\frac{5-2\al}{14-2\al}\right)^2$ \\
& & (0)\ $2\cdot2\cdot\frac{5-2\al}{14-2\al}$\\
\hline
3 & (2)\ $(5-\al)\left(\frac{9}{14-2\al}\right)^2$ & (0)\ $2\cdot 
\frac{9}{14-2\al}$ \\
&& (2)\ $(5-\al)\cdot \frac{9}{14-2\al}\cdot \frac{5-2\al}{14-2\al}$\\
\hline
\end{tabular}
\end{center}
}

It is not difficult to see that for $\al<1$, we have 
$
\prob[\OX= \OY]>1/2
$
and that, given that $\OX=\OY$ holds, Eve has no information at all about
what this bit is. This means that  the repeat-code protocol 
mentioned in Example~1
allows for classical key agreement in this situation~\cite{ka},\, \cite{diss}. 
For $\al\geq 1$, classical key agreement, like quantum key
agreement, seems impossible however.
We will discuss this further in Section~\ref{secbii}.
The results of Example~3 are illustrated in Figure~\ref{qcw}.
\begin{figure}[!h]
\hbox{
\centerline{
\psfig{figure=qcw.eps,width=8cm}}}
\caption{The Results of Example 3}
\label{qcw}
\end{figure}
\exend
\\ \ 




\section{Bound Intrinsic Information and Binarizations}\label{secbii}

Conjecture~\ref{conprot} suggests that, in contrast to previous 
beliefs in classical information theory, bound entanglement has a 
classical counterpart, which we call {\em bound information}.


\begin{defi}
{\rm
Let $P_{XYZ}$ be a  distribution with $\ida>0$. Then if $S(X;Y||Z)>0$
holds for this distribution, the intrinsic information between $X$ and $Y$,
given $Z$, is called {\em free}. Otherwise, if $S(X;Y||Z)=0$, the 
intrinsic information is called {\em bound}.
}
\defend
\end{defi}


We are now interested in a proof of the existence of such bound information.
In view of the fact that Conjecture~\ref{conprot} might be hard to prove 
in general, it is worth to look at a classical ``translation'' of a 
bound entangled quantum state directly and analyze it with the tools 
of classical information theory. 

This analysis shows that an important concept 
in the context of key agreement from classical information are so-called
{\em binarizations}. We give evidence for the fact that classical 
information can be used for key agreement only if the random variables 
$X$ and $Y$ can be made binary by local operations (i.e., by sending 
them over some binary-output channel) in such a way that the resulting 
{\em binary\/} random variables still share some information. The quantum
counterpart of this insight may result in an easy characterization 
and better understanding of the strange phenomenon of bound entanglement.

Let is look at the states $\Psi_{\al}$ of Example~3 again. 
First, we prove that whenever the random variable $Y$ is ``binarized,''
i.e., sent through a binary-output channel $P_{\OY|Y}$ (or a ternary-output
channel but where only two symbols are actually
 considered in the computation of the mutual information), then the intrinsic 
information vanishes (Proposition~\ref{prop1}).

Proposition~\ref{prop2}
 on the other hand suggests that intrinsic information which 
does not resist any binarization must be bound: Whenever secret-key agreement 
is possible with $X$ and  $Y$ and with respect to $Z$, then there exist 
binarizations of 
a certain number of 
repetitions of $X$ and $Y$ such that the 
intrinsic information remains positive. 




\begin{prop}\label{prop1}
Assume  the  distribution of Example~3 with
 $\al\in[1,2)$.
Let $P_{\OY|Y}$ be an arbitrary conditional distribution with
$\overline{\CY}=\{0,1,\gd\}$, and let 
$\CE$ be the event that $\OY\in\{0,1\}$.
Then $I(X;\OY\downarrow Z\, |\, \CE)=0$.
\end{prop}

\proof
We only have to consider the case $\alpha=1$. This implies the statement for 
all $\al\in[1,2)$. Let the following channel $P_{\OY|Y}$ be given (where
$\overline{\CY}=\{0,1,\gd\}$):
\begin{eqnarray*}
P_{\OY|Y}(0,1)&=&x\ ,\ \ P_{\OY|Y}(0,2)\ =\ y\ ,\ \ P_{\OY|Y}(0,3)\ =\ z\ ,\\
P_{\OY|Y}(1,1)&=&u\ ,\ \ P_{\OY|Y}(1,2)\ =\ v\ ,\ \ P_{\OY|Y}(1,3)\ =\ w\ .\\
\end{eqnarray*}
Here, we have $x,y,z,u,v,w,x+u,y+v,z+w\in[0,1]$. We get the 
following distribution $P_{X\OY Z|\CE}$ (to be normalized).
\\
\begin{center}
\begin{tabular}{|c||c|c|c|}
\hline
\ \ $X$ & $1$ & $2$ & $3$\\
$\OY$ ($Z$) &&&\\
\hline\hline
 & (0)\ $2x$ & (0)\ $2y$ & (0)\ $2z$\\
0 & (3)\ $y$ & (1)\ $4x$ & (2)\ $x$\\
  & (5)\ $4z$ & (6)\ $z$ & (4)\ $4y$\\
\hline
& (0)\ $2u$ & (0)\ $2v$ & (0)\ $2w$\\
1 & (3)\ $v$ & (1)\ $4u$ & (2)\ $u$\\
  & (5)\ $4w$ & (6)\ $w$ & (4)\ $4v$\\
\hline
\end{tabular}
\end{center}




The only symbol $z$ of $Z$ for which $I(X;\OY|Z=z,\CE)>0$ holds is $z=0$.
Let us now consider a channel $P_{\OZ|Z}$ with $\overline{\CZ}=\{\overline{0}, 
\overline{1}, \overline{2}, \overline{3}, \overline{4}, \overline{5}, 
\overline{6}\}$ and $P_{\OZ|Z}(0,0)=1$. Furthermore,
 $P_{\OZ|Z}(\overline{0},1)=c$ and $P_{\OZ|Z}(\overline{1},1)=1-c$,
and analogously for $Z=2,3,4,5$, and $6$
with transition
probabilities $e,a,f,b$, and $d$, respectively. Then we get for the 
column vectors of the $P_{X\OY|\OZ=\overline{0}}$ matrix:
\[
\left[
2{x \choose u}+a{y \choose v}+4b{z \choose w}\, ,\,
 4c{x \choose u}+2{y \choose 
v}+d{z \choose w}\, ,\, e{x \choose u}+4f{y \choose v}+2{z \choose w}
\right].
\]
Clearly, the three vectors are linearly dependent. We can assume  that 
\[
{x \choose u}=\lambda_1{y \choose v}+\lambda_2{z \choose w}
\]
holds for some $\lambda_1,\lambda_2\in[0,\infty)$.
(The other cases are analogous.) 

Let $\vec{s}:={y \choose v}$ and $\vec{t}:={z \choose w}$.
We then get for the above matrix 
\[
\left[
(a+2\lambda_1)\vec{s}+(4b+2\lambda_2)\vec{t}, 
(2+4c\lambda_1)\vec{s}+(d+4c\lambda_2)\vec{t},
(4f+e\lambda_1)\vec{s}+(2+e\lambda_2)\vec{t}\, 
\right].
\]
The corresponding distribution satisfies $I(X;\OY|\OZ=\overline{0},\CE)=0$
if 
\[
\frac{a+2\lambda_1}{4b+2\lambda_2}=\frac{2+4c\lambda_1}{d+4c\lambda_2}
=\frac{4f+e\lambda_1}{2+e\lambda_2}
\]
holds. This is equivalent  to 
\begin{eqnarray*}
\lambda_1(2d-16bc)+\lambda_2(4ac-4) & = & 8b-ad\ ,\\
\lambda_1(4-4be)+\lambda_2(ae-8f) & = & 16bf-2a\ .
\end{eqnarray*}
%
We show that this system is solvable, with $(a,b,c,d,e,f)\in[0,1]^6$, for all 
$\lambda_1,\lambda_2\in[0,\infty)$. For this, we prove that for all
sufficiently large numbers $R>0$, the equations are solvable 
for all pairs $(\lambda_1,\lambda_2)$
on the path $(0,0)$-$(R,0)$-$(R,R)$-$(0,R)$-$(0,0)$, 
and that the corresponding path in $[0,1]^6$ is homeomorphic to $S^1$.
Then the claim follows by a simple topological argument.

We only sketch the remainder of the proof.
For $(\lambda_1,\lambda_2)=(0,0)$, the equations are solvable by setting
\begin{equation}\label{nullnull}
d=f=1\mbox{\ \ and\ \ }8b=d\ .
\end{equation}
For $(\lambda_1,\lambda_2)=(R,0)$, where we assume $R$ to be sufficiently
 large, a solution is given by
\[
b\approx e\approx 1\mbox{\ \ and\ \ }d\approx 8c 
\] 
(where additionally both equations of (\ref{nullnull}) should {\em not\/} 
be satisfied nor approximately satisfied).
For $(\lambda_1,\lambda_2)=(R,R)$, the equalities are 
\begin{eqnarray*}
R(2d-16bc+4ac-4) & = & 8b-ad\\
R(4-4be+ae-8f) & = & 16bf-2a
\end{eqnarray*}
with a possible approximate solution
\[
b\approx e\approx 0\ ,\ \ c\approx d\approx 1\ ,\ \ a\approx f\approx 1/2\ .
\]
Finally, the case $(\lambda_1,\lambda_2)=(0,R)$ can be  solved by
\[
a\approx c\approx 1\mbox{\ \ and\ \ }e\approx 8f\ .
\]
When combining the solutions for the different cases, it is not difficult 
to see that there exists a path
$\gamma$ in $[0,1]^6$ that, mapped to the $(\lambda_1,\lambda_2)$ plane,
exactly corresponds to the square $(0,0)$-$(R,0)$-$(R,R)$-$(0,R)$-$(0,0)$.
This is true for all sufficiently large 
$R$, and thus the argument is finished.
\peon



\begin{prop}\label{prop2}
  Let $X$, $Y$ and $Z$ be random variables with $S(X;Y\|Z)>0$.
  Then for each $\varepsilon>0$ there exist a number $N$ and
  ternary-output channels $P_{\overline{X}|X^N}$ and $P_{\overline{Y}|Y^N}$ with
  ranges $\overline{\mathcal{X}} = \overline{\mathcal{Y}} = \{0, 1, \Delta\}$
  such that
  \begin{eqnarray}
{\rm Prob}[E'] & > & 0\\
    P[\overline{X} = \overline{Y} | E] & > & 1-\varepsilon \label{eq:skcond1} \\
    P[\overline{X} = 0 | E'] = P[\overline{X} = 1 | E'] & = & 1/2 \label{eq:skcond2}\\
    H(\overline{X} | Z^N, E) & > & 1-\varepsilon \label{eq:skcond3}
  \end{eqnarray}
  where $E$ and $E'$ are the events defined by $\overline{X} \neq \Delta
  \neq \overline{Y}$ and $\overline{X} = \overline{Y} \neq \Delta$, respectively (note
  that $E' = E \cap [\overline{X} = \overline{Y}]$). In particular, we have 
$I(\OX;\OY\downarrow Z^N,E)>0$.
\end{prop}



\proof
  According to the definition of the secret-key rate, for each
  $\varepsilon'$ there exist a number $N$ and a protocol that allows
  Alice and Bob for computing keys $S_A, S_B \in \{0,1\}^K$ out of
  $N$ realizations of the random variables $X$ and $Y$ such that
  \begin{eqnarray}
    P[S_A \neq S_B] & < & \varepsilon' \label{eq:skorg1} \\
    H(S_A | Z^N C) & > & K - \varepsilon' \label{eq:skorg2}
  \end{eqnarray}
  where $C$ is the communication exchanged over the public channel.
  Let  $S'_A$ and $S'_B$ to be the
  first bit of $S_A$ and $S_B$, respectively. It is clear from
  (\ref{eq:skorg1}) that
  \begin{equation} \label{eq:bcond1}
    P[S'_A \neq S'_B] < \varepsilon' 
  \end{equation}
  and from (\ref{eq:skorg2})
  \begin{equation} \label{eq:bcond2}
    H(S'_A | Z^N C) = H(S_A | Z^N C) - H(S_A | S'_A Z^N C) 
      > K - \varepsilon' - (K-1) = 1- \varepsilon'.
  \end{equation}
  Define the functions
  \begin{equation}
    e: c \mapsto P[S'_A \neq S'_B | C=c]
  \end{equation}
  and 
  \begin{equation}
    k:  c \mapsto 1- H(S'_A | Z^N, C=c).
  \end{equation}
  From conditions (\ref{eq:bcond1}) and (\ref{eq:bcond2}) we have that
  $E_C[e(C)] < \varepsilon'$ and $E_C[k(C)] < \varepsilon'$ (where
  $E_C$ is the expectation value over all possible communications $c \in
  \mathcal{C}$). Since both $e$ and $k$ only take on positive values,
  it follows immediately that $P[e(C) < 2\varepsilon']\geq 1/2$ and
  $P[k(C) < 2\varepsilon']\geq 1/2$, which implies that there exists a
  particular communication string 
$c \in \mathcal{C}$ such that $e(c) < 2\varepsilon'$
  and $k(c) < 2\varepsilon'$, or
  \begin{eqnarray} 
    P[S'_A \neq S'_B | C=c] & < & 2 \varepsilon' \label{eq:optc1} \\
    H(S'_A | Z^N, C=c) & > & 1-2\varepsilon' \label{eq:optc2}.
  \end{eqnarray}
  
  In general, a secret-key agreement protocol consists of $2M$ steps
  (where $M$ is itself a random variable) such that in each step Alice
  sends the information $C_i$ to Bob (for $i$ odd) or Bob sends $C_i$
  to Alice (for $i$ even). After this communication phase, Alice and
  Bob compute their secret-key bits $S'_A$ and $S'_B$, respectively. We
  thus have
  \begin{eqnarray*}
      P_{S'_AS'_BC|X^NY^N}
 &   = &
      P_{S'_A|X^N C} \cdot P_{S'_B|Y^N C} \cdot 
      P_{C_{2M}|C^{2M-1} Y^N} \cdot P_{C_{2M-1}|C^{2M-2} X^N}\cdot  \\
      && \qquad \ldots\cdot 
      P_{C_2|C_1 Y^N} \cdot P_{C_1|X^N}
  \end{eqnarray*}
  (the arguments are omitted in this expression), where $C^i := C_1
  \cdots C_i$ and $C := C^{2M}$. When the terms are rearranged, 
this expression
  can be written as
  \begin{equation}
      P_{S'_AS'_BC|X^NY^N}(s_A, s_B, c, x, y) 
    = 
      p_A(s_A, c, x) \cdot p_B(s_B, c, y) 
  \end{equation}
  for appropriate functions $p_A$ and $p_B$.  For a communication string $c \in
  \mathcal{C}$ satisfying (\ref{eq:optc1}) and (\ref{eq:optc2}), we define
  \begin{eqnarray}
    P_{\overline{X}'|X^N} (\overline{x}', x) := p_A(\overline{x}', c, x) \\
    P_{\overline{Y}'|Y^N} (\overline{y}', y) := p_B(\overline{y}', c, y)
  \end{eqnarray}
  for all $\overline{x}', \overline{y}' \in \{0, 1\}$, $x \in \mathcal{X}^N$,
  $y \in \mathcal{Y}^N$. (Note that this completely determines the
  channels $P_{\overline{X}'|X^N}$ and $P_{\overline{Y}'|Y^N}$.)  Then we get 
from 
  (\ref{eq:optc1}) and (\ref{eq:optc2})
  \begin{eqnarray} 
    P[\overline{X}' \neq \overline{Y}' | \overline{X}' \neq \Delta \neq \overline{Y}'] 
      & < & 2 \varepsilon' \label{eq:bitp1} \\
    H(\overline{X}' | Z^N, \overline{X}' \neq \Delta \neq \overline{Y}') 
      & > & 1-2\varepsilon' \label{eq:bitp2}.
  \end{eqnarray}
It remains to show that equality (\ref{eq:skcond2}) holds.  Let
  therefore 
  \begin{equation}
      \delta
    := 
      1/2 - P[\overline{X}' = 0|\overline{X}' = \overline{Y}' \neq \Delta]
  \end{equation}
  and assume without loss of generality that $\delta \geq 0$. Define
  \begin{equation}
    P_{\overline{X}|\overline{X}'} (\overline{x}, \overline{x}') = 
    \begin{cases}
      1         & \text{if $\overline{x} = \overline{x}' = \Delta$ 
                    or $\overline{x}=\overline{x}'=0$} \\
      \frac{1/2-\delta}{1/2+\delta} 
                & \text{if $\overline{x} = \overline{x}' = 1$} \\
      1 - \frac{1/2-\delta}{1/2+\delta} 
                & \text{if $\overline{x} = \Delta$ and $\overline{x}' = 1$} \\
      0         & \text{otherwise}
    \end{cases}
  \end{equation}
  and $\overline{Y} := \overline{Y}'$. Then
  \begin{equation}
     P[\overline{X} = 0 | E'] = P[\overline{X} = 1 | E'] = 1/2.
  \end{equation}
  It can easily be verified from (\ref{eq:bitp2}) that $\delta$ is of
  order $\varepsilon'$ and thus the assertion follows
  from (\ref{eq:bitp1}) and (\ref{eq:bitp2}).
\pe
Propositions~\ref{prop1} and~\ref{prop2} 
do not  imply that Alice and Bob share 
bound information in the considered distribution. More precisely, 
the following statement, which we give as a conjecture, is the missing 
gap in the way towards proving the existence  of bound information.

\begin{con}\label{gapcon}
Let $P_{XYZ}$ be a distribution. Then there exist binary-output 
channels $P_{\OX|X}$ and $P_{\OY|Y}$ with $I(\OX;\OY\downarrow Z)>0$
if and only if there exist, for some $N$, binary-output channels 
$P_{\overline{X}|X^N}$ and $P_{\overline{Y}|Y^N}$ such that 
$I(\overline{X};\overline{Y}\downarrow Z^N)>0$ holds.
\end{con}


The results of this section suggest that free intrinsic information
can be binarized, whereas bound information cannot. We finally conjecture
that this is also a way of distinguishing free from bound entanglement 
on the quantum side.

\begin{con}
Let $\rho_{AB}$ be a mixed state over $\HA\otimes \HB$. Then $\rho_{AB}$
 is free entangled if and only if 
there are two-dimensional projectors $P_A$ and $P_B$ such that 
$(P_A\otimes P_B)\rho_{AB} (P_A\otimes P_B)$ 
is an entangled two-Qbit state. 
\end{con}

Note that it is clear that if Alice and Bob can produce entangled 
Qbits, then they can always purify the original state because 
these Qbits are free entangled. The conjecture states that the
reverse implication 
is also true. 
This would mean that $\rho_{AB}$ is free entangled if and only if  Alice and
Bob can produce entangled Qbits using a single copy of $\rho_{AB}$.







\section{Concluding Remarks}
We have considered the model of information-theoretic key agreement 
by public discussion from correlated information. More precisely, 
we have compared  scenarios where the joint information is 
given by classical random variables and by quantum states (e.g., after 
execution of a quantum protocol). We proved a close connection between
such classical and quantum information,  namely between
intrinsic information and entanglement.


Furthermore,
 examples have been presented that provide 
 evidence for the fact that the close connections between classical and quantum 
information extend to the level of the 
protocols. A consequence would be that  the powerful tools and statements 
on the existence or rather {\em non-existence\/} of quantum-purification
protocols immediately carry over to the classical scenario, where it 
is often  unclear how 
to show that no protocol exists.
Many examples
coming from measuring bound entangled states, and for which none 
of the known classical secret-key agreement protocols is successful,
as well as some general facts on binarizing classical information,
strongly
suggest that bound entanglement  has a classical counterpart:
intrinsic information which cannot be distilled to a secret key.
This stands in sharp contrast to what was previously believed about 
classical key agreement.
This is one of the rare examples for which 
 a concept of 
information processing is first discovered on the quantum domain
and then leads to new insight in the classical regime.



Finally, we have proposed a measure for entanglement, based on classical 
information theory, with all  the properties required for such a measure.


\section*{Acknowledgments}

The authors thank Gilles Brassard, Nicolas Cerf, Claude Cr\'epeau, 
Serge Massar, 
Ueli Maurer,  Michele Mosca, 
Sandu Popescu, and Alain Tapp for interesting discussions. 
The first and the third author were partially supported by 
Switzerland's SNF,  and the third author was partially supported by 
 Canada's NSERC.



%\newpage


\begin{thebibliography}{10}

\bibitem{bg}
H.~Bechmann-Pasquinucci and N.~Gisin,
Incoherent and coherent eavesdropping in the six-state protocol
of quantum cryptography, {\em Phys.\ Rev.~A}, Vol.~59, No.~6,
pp.~4238--4248, 1999.

\bibitem{benal}
       C.~H.~Bennett, G.~Brassard, S.~Popescu,
B.~Schumacher, J.~A.~Smolin, and W.~K.~Wooters,
Purification of noisy entanglement and faithful teleportation
via noisy channels,
{\em Phys.\ Rev.\ Lett.}, Vol.~76, pp.~722--725, 1996.

\bibitem{BB84} 
C.~H.~Bennett and G.\ Brassard, 
Quantum cryptography: public key distribution and coin tossing,
{\it Proceedings of
the IEEE International Conference on Computer, Systems, and Signal
Processing}, IEEE, pp.~175--179, 1984.


\bibitem{Dagmar6state} 
D.\ Bruss, Optimal eavesdropping in quantum cryptography
with six states, {\em Phys.\ Rev.\ Lett.}, Vol.~81, No.~14, pp.~3018--3021, 1998.

\bibitem{Buzek} 
V.\ Bu\v{z}ek and M.\ Hillery, 
Quantum copying: beyond the no-cloning theorem,
{\em Phys.\ Rev.~A}, Vol.\ 54, pp.~1844--1852, 1996.
       


\bibitem{CHSH}
J.\, F.~Clauser, M.\, A.~Horne, A.~Shimony and R.\, A.\ Holt,
Proposed experiment to test local hidden-variable theories,
{\it Phys. Rev. Lett.}, Vol.~23, pp.~880--884, 1969. 



\bibitem{csikor78} I.~Csisz\'{a}r and J.~K\"orner, 
Broadcast channels with confidential messages, 
{\em IEEE Transactions on Information Theory\/}, 
Vol.~IT-24, pp.~339--348, 1978.


\bibitem{QPA} D.\ Deutsch, A.\ Ekert, R.\ Jozsa, C.\ Macchiavello,
S.\ Popescu, and A.\ Sanpera, Quantum privacy amplification and the 
security of 
quantum cryptography over
noisy channels, {\em Phys.\ Rev.\ Lett.}, Vol.~77, pp.~2818--2821, 1996.


\bibitem{difhel76} W.~Diffie and M.~E.~Hellman, New directions in
  cryptography, {\em IEEE Transactions on Information Theory}, 
Vol.~22, No.~6, pp.~644--654, 1976.




\bibitem{benpt}
D.~P.~DiVincenzo, P.~W.~Shor, J.~A.~Smolin, B.~M.~Terhal,
and A.~V.~Thapliyal, Evidence for bound entangled states with negative 
partial transpose, quant-ph/9910026, 1999.

\bibitem{ekert}
A.\ E.\ Ekert, Quantum cryptography based on Bell's theorem,
{\em Phys.\ Rev.\ Lett.}, Vol.~67, pp.~661--663, 1991. See also  {\em Physics World},
March 1998.

\bibitem{FGGNP} C.~Fuchs, N.~Gisin, R.~B.~Griffiths, C.~S.~Niu, and A.~Peres,
Optimal eavesdropping in quantum cryptography -- I: information bound and optimal
strategy, {\em 
           Phys.\ Rev.~A},  Vol.~56, pp.~1163--1172, 1997.


\bibitem{gisin89} N.~Gisin, Stochastic quantum dynamics and relativity, 
{\em Helv.\ Phys.\ Acta}, Vol.~62, pp.~363--371, 
1989.



\bibitem{GisinHuttner97} N.\ Gisin and B.\ Huttner, 
Quantum cloning, eavesdropping, and Bell inequality,
{\em Phys.\ Lett.~A}, Vol.~228,
pp.~13--21, 1997.

\bibitem{gismas} N.~Gisin and S.~Massar, 
Optimal quantum cloning machines,
{\em Phys.\ Rev.\ Lett.}, Vol.~79, pp.~2153--2156, 1997.

\bibitem{grw00}
N.\ Gisin, R.\ Renner, and S.\ Wolf,
Bound information: the classical analog to bound entanglement,
in {\em Proceedings of 3ecm}, Birkh\"auser Verlag, 2000.



\bibitem{giswol00}
N.\ Gisin and S.\ Wolf,
Linking classical and quantum key agreement: is there ``bound
information''?, in {\em Proceedings of CRYPTO 2000},  
Lecture Notes in Computer Science, vol.~1880, pp.~482--500,
Springer-Verlag, 2000.

\bibitem{giswol99} N.~Gisin and S.~Wolf, Quantum cryptography on 
noisy channels: quantum versus classical key agreement protocols,
{\em Phys.\ Rev.\ Lett.}, Vol.~83, pp.~4200--4203, 1999.





\bibitem{Horodecki}
       M.~Horodecki, P.~Horodecki, and  R.~Horodecki, 
Mixed-state entanglement and distillation: is there a ``bound''
entanglement in nature?, {\em 
Phys.\ Rev.\ Lett.}, Vol.~80, pp.~5239--5242, 1998.



\bibitem{hohoho97}
M.~Horodecki, P.~Horodecki, and  R.~Horodecki, 
Inseparable 2 spin $1/2$ density matrices can be distilled to a singlet 
form, {\em Phys.\ Rev.\ Lett.}, Vol.~78, p.~574, 1997.




\bibitem{HorodeckiPartialTransp} 
P.~Horodecki, Separability criterion and inseparable mixed states 
with positive partial transposition, 
{\em Phys.\ Lett.~A}, Vol.~232, p.~333, 1997.

\bibitem{dreih}
       P.~Horodecki, M.~Horodecki, and  R.~Horodecki, 
Bound entanglement can be activated, 
{\em 
Phys.\ Rev.\ Lett.}, Vol.~82, pp.~1056--1059, 1999.
quant-ph/9806058.


\bibitem{hjw93} L.\, P.\ Hughston, R.\ Jozsa, and
W.\, K.\ Wootters, 
A complete classification of quantum ensembles having
      a given density matrix,
{\em Phys.\ Lett.~A}, Vol.~183, pp.~14--18, 1993.


\bibitem{landau98}
R.\ Landauer, Information is inevitably physical, {\em Feynman and 
Computation 2}, Addison Wesley, Reading, 1998.

\bibitem{landau96}
R.\ Landauer, The  physical nature of information, {\em Phys.\ Lett.\ A},
Vol.~217, p.~188, 1996.



\bibitem{ka} U.~Maurer, Secret key agreement by public discussion from
common information, {\em IEEE Transactions on Information Theory\/},
Vol.~39, No.~3, pp.~733--742, 1993.

\bibitem{strong}
U.~Maurer and S.~Wolf,
Information-theoretic key agreement: from weak to strong secrecy for free,
{\em Proceedings of EUROCRYPT 2000},
Lecture Notes in Computer Science, Vol.~1807, 
pp.~352--368, Springer-Verlag, 2000.

\bibitem{ittrans} 
U.~Maurer and S.~Wolf, Unconditionally secure key agreement 
and the  intrinsic
conditional  information, 
{\em IEEE Transactions on Information Theory\/}, 
Vol.~45, No.~2, pp.~499--514, 1999.






\bibitem{qp9609013} 
N.\ D.\ Mermin, The Ithaca interpretation
of quantum mechanics,
{\em Pramana}, Vol.~51, pp.~549--565, 1998.




\bibitem{Peresbook} A.~Peres, {\em Quantum theory: concepts and methods}, 
Kluwer Academic Publishers, 1993.

\bibitem{Peres} A.~Peres, 
Separability criterion for density matrices,
{\em Phys. Rev. Lett.}, Vol.~77, pp.~1413--1415, 1996.


\bibitem{qp9610044} 
S.~Popescu and D.\ Rohrlich, Thermodynamics and the measure of 
entanglement,  quant-ph/9610044, 1996.

\bibitem{ribo98} G.\ Ribordy,
J.~D.~Gautier, N.~Gisin, O.~Guinnard, and H.~Zbinden, Automated plug and play quantum 
key distribution,  {\em Electron.\ Lett.},
Vol.~34, pp.~2116--2117, 1998.

\bibitem{shannon}
C.~E.~Shannon, Communication theory of secrecy systems,
{\em Bell System Technical Journal}, Vol.~28, pp.~656--715, 1949.

\bibitem{vernam26} 
G.~S.~Vernam, Cipher printing telegraph systems for secret wire and 
radio telegraphic communications, {\em Journal of the American
Institute for Electrical Engineers}, Vol.~55, pp.~109--115, 1926.


\bibitem{diss} S. Wolf, {\em Information-theoretically and computationally
secure key agreement in cryptography}, ETH dissertation No.\ 13138, 
Swiss Federal Institute of Technology (ETH Zurich), May 1999.


\bibitem{wyner75}
A.~D.~Wyner, The wire-tap channel,
{\em Bell System Technical Journal}, Vol.~54, No.~8, pp.~1355--1387, 1975.

\bibitem{QCexp} H.~Zbinden, H.~Bechmann, G.~Ribordy, and N.~Gisin, Quantum cryptography, 
{\em Applied Physics B}, Vol.~67, pp.~743--748, 
1998.








\end{thebibliography}

\end{document}

\newpage
\section*{Appendix A: Figures}


\begin{figure}[h]
\hbox{
\centerline{
\psfig{figure=../key/aarhus/ska.eps,width=6cm}}}
\caption{Secret-Key Agreement from Common Information}
\label{maumod}
\end{figure}




\begin{figure}[h]
\hbox{
\centerline{
\psfig{figure=channel.eps,width=4.5cm}}}
\caption{The Channel $P_{\OZ|Z}$ in Example~1}
\label{channel}
\end{figure}

\nopagebreak

\begin{figure}[!h]
\hbox{
\centerline{
\psfig{figure=qcw.eps,width=9cm}}}
\caption{The Results of Example 3}
\label{qcw}
\end{figure}


\newpage
\section*{Appendix B: Measuring in ``bad'' Bases}

In this appendix we show, by two examples, that the statements of Theorems~1 and~2
do not always hold for the standard bases and, in particular, not for 
arbitrary bases: Alice and Bob as well as Eve can perform bad measurements 
and give away an initial advantage. Let us begin with an example where 
measuring in the standard basis is a bad choice for Eve.
\\ \

\noi
{\it Example 6.}
Let us consider the quantum states
\[
\Psi  =  \frac{1}{\sqrt{5}}\, (|00+01+10\rangle \otimes|0\rangle  + |00+11\rangle \otimes|1\rangle )\ ,\ \ 
\rho_{AB}  =  \frac{3}{5}\,  P_{|00+01+10\rangle } + \frac{2}{5}\,  P_{|00+11\rangle }\ .
\]
If Alice, Bob, and Eve measure in the standard bases, we get the classical distribution 
 (to be normalized)
\begin{center}
\begin{tabular}{|c||c|c|}
\hline
\ \ X & 0 & 1\\
Y (Z) &&\\
\hline\hline
0 & (0)\ 1 & (0)\ 1 \\
 &  (1)\ 1 & (1)\ 0 \\
\hline
1 & (0)\ 1 & (0)\ 0 \\
  & (1)\ 0 & (1)\ 1\\
\hline
\end{tabular}
\end{center}
For this distribution, $\ida>0$ holds. Indeed, 
even $S(X;Y||Z)>0$ holds.
This is not surprising since both $X$ and $Y$ 
are binary, and since the described parallels suggest that in this 
case, positive intrinsic information implies that a secret-key 
agreement protocol exists. 

The proof of $S(X;Y||Z)>0$ in this situation is analogous to the proof of this 
fact in Example~3. The protocol consists of Alice and Bob independently 
making their bits symmetric. Then the repeat-code protocol can be 
applied. 

However, the 
partial-transpose condition shows that $\rho_{AB}$ is separable. This 
means that measuring in the standard basis is bad for Eve. Indeed,
let us rewrite $\Psi$ and $\rho_{AB}$ as
\begin{eqnarray*}
\Psi   & = &  \sqrt{\Lambda}\, |m,m\rangle \otimes|\tilde 0\rangle + 
     \sqrt{1-\Lambda}\, |-m,-m\rangle \otimes|\tilde 1\rangle\ ,\\
\rho_{AB} &  = &  \frac{5+\sqrt{5}}{10}\,  P_{|m,m\rangle } + 
         \frac{5-\sqrt{5}}{10}\,  P_{|-m,-m\rangle }\ ,
\end{eqnarray*}
where
$\Lambda  =  (5+\sqrt{5})/10$, 
$|m,m\rangle  =  |m\rangle \otimes|m\rangle$,
$|\pm m\rangle  =  \sqrt{(1\pm\eta)/2}\, |0\rangle  \pm \sqrt{(1\mp\eta)/2}\, |1\rangle$, and 
$\eta  =  1/\sqrt{5}$.

In this representation, $\rho_{AB}$ is obviously separable. It also means that 
 Eve's optimal measurement basis is
\[
|\tilde 0\rangle =\sqrt{\Lambda}\, |0\rangle  - \frac{1}{\sqrt{5\Lambda}}\, |1\rangle \ ,\ \ 
|\tilde 1\rangle =-\sqrt{1-\Lambda}\, |0\rangle  - \frac{1}{\sqrt{5(1-\Lambda)}}\, |1\rangle\ . 
\]
Then, $\ida=0$ holds for the resulting classical distribution.
\exend
\ \\

\noi
Not surprisingly, there also exist examples of distributions for which measuring 
in the standard bases is bad for Alice and Bob. These states are entangled, but $\ida=0$
holds.
\\ \

\noi
{\it Example 7.}
Let the following classical distribution be given:
\begin{center}
\begin{tabular}{|c||c|c|}
\hline
\ \ X & 0 & 1\\
Y (Z) &&\\
\hline\hline
0 & (0)\ 0.0082 & (0)\ 0.0219 \\
 &  (1)\ 0.0006 & (1)\ 0.0202 \\
\hline
1 & (0)\ 0.0729 & (0)\ 0.3928 \\
  & (1)\ 0.0905 & (1)\ 0.3889204545\\
\hline
\end{tabular}
\end{center}

Because of 
\[
(0.0082+0.0006)\cdot (0.03928+0.3889204545)=(0.0219+0.0202)\cdot (0.0729+0.0905)
\]
we have $I(X;Y)=0$, thus $\ida=0$.
On the other hand, the corresponding quantum state,
for which the above distribution results by measuring in the standard bases,
can be shown to be entangled. 
\exend









\section*{Appendix C: A Protocol for Advantage Distillation}


 The following protocol for classical advantage distillation is called 
  {\em repeat-code protocol\/} and was first proposed in~\cite{ka}. 
Note that there exist  more 
efficient protocols in terms of the amount of extractable secret key.
However, since we only want to prove  qualitative possibility results,
 it is sufficient to look at this simpler protocol.
Assume the scenario of Example~1.


Let $N>0$ be an even integer, and let Alice choose a random bit $C$ and send
the block
\[
X^N\oplus C^N:=[X_1\oplus C,X_2\oplus C,\ldots, X_N\oplus C]
\]
over the classical channel.
Here, $X^N$ stands for the block $[X_1,X_2,\ldots, X_N]$ of $N$ consecutive
realizations of the random variable $X$,
whereas $C^N$ stands for the $N$-bit block $[C,C,\ldots,C]$.
Bob then computes
$[(C\op X_1)\op Y_1,\ldots,(C\op X_N)\op Y_N]$ and (publicly)
accepts exactly if this block
is equal to either $[0,0,\ldots,0]$ or $[1,1,\ldots,1]$. In other words,
Alice and Bob make use of a repeat code of length $N$ with the only two
 codewords
$[0,0,\ldots,0]$ and $[1,1,\ldots,1]$.


Bob's conditional error
probability $\be_N$
when guessing the bit sent by Alice, given that he accepts, is
\[
\be_N=\frac{1}{p_{a,N}}\cdot D^N\leq\left(\frac{D}{1-D}\right)^N\ ,
\]
where $p_{a,N}=D^N+(1-D)^N$ 
is the probability that Bob accepts the received 
block.
It is obvious that Eve's optimal 
strategy for guessing $C$  is to compute the block
$[(C\oplus X_1)\oplus Z_1,\ldots,(C\oplus X_N)\oplus Z_N]$ and guess $C$ as
$0$ if at least half of the bits in this block are $0$, and as $1$ otherwise.
Given that Bob correctly accepts, Eve's error probability when guessing the
bit $C$ with the optimal strategy
is lower bounded by $1/2$ times the probability that she decodes
to a block with $N/2$ bits $0$ and the same number of $1$'s. Hence
we get that 
\[
\ga_N\geq \frac{1}{2}{N \choose N/2}(1-\de)^{N/2}\de^{N/2}
\geq\frac{K}{\sqrt{N}}\cdot\left(2\sqrt{(1-\de)\de}\right)^N
\] 
holds for some constant $K$ and 
for sufficiently large $N$ by using Stirling's formula.
Note that Eve's error probability given that Bob accepts is 
asymptotically equal to her error probability given that 
Bob {\em correctly\/} accepts because Bob accepts erroneously
only with asymptotically vanishing probability, given that he 
accepts.

Although it is not the adversary's ultimate goal to guess the bits 
$C$ sent by Alice, it has been shown 
that the fact that $\be_N$ decreases exponentially faster than $\ga_N$
implies that for sufficiently large $N$, Bob has more (Shannon) information
about the bit $C$ than Eve (see for example~\cite{ittrans}). 
Hence Alice and Bob have managed to generate
new random variables with the property that Bob obtains more information about 
Alice's random bit than Eve has. Thus $S(X;Y||Z)>0$ holds.



\end{document}





































































