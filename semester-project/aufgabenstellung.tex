\documentclass[E]{DASA}
\renewcommand{\asst}{Martin Hirt}

\begin{document}

\SA{October 23, 2001}{February 15, 2002}{Juraj Skripsky}%
{Minimal Models for Receipt-Free Voting}

\intro

Secret-ballot voting schemes belong to the most significant applications of
cryptographic protocols. They allow the computation of the tally of the
cast votes, while preserving privacy of each particular ballot. More
precisely, the following properties must be satisfied:
\begin{itemize}\itemsep=0pt\parsep=0pt
\item {\sc Secrecy.} It is infeasible to find out which voter has submitted
  which vote.
\item {\sc Eligibility.} Only entitled voters are able to submit a vote.
\item {\sc no double-voting.} Every entitled voter can cast only one single
  vote.
\item {\sc Validity.} Only valid votes are counted, e.g., ``yes'' and
  ``no'' votes.
\item {\sc Correctness.} The published tally is the correct sum of all
  valid votes.
\item {\sc Local verifiability.} Every voter can verify whether his vote
  is included in the tally.
\item {\sc Global verifiability.} Anyone can verify the correctness of the
  tally.
\end{itemize}

An important concept that was neglected in the classical literature is
{\em receipt-freeness}. The goal of receipt-freeness is to thwart
vote-selling and coercion. More formally, receipt-freeness requires that
the voter is not able to construct a receipt (a witness) of the cast vote.

The concept of receipt-freeness was introduced by Benaloh and Tuinstra
\cite{BenTui94}. Later, many receipt-free protocols were proposed
\cite{SakKil95,Okamot96,Okamot97,HirSak00,LeeKim00,Hirt01,BPPSF01}.  All
these protocols make additional (rather unrealistic) assumptions on the
communication model, e.g., existence of a voting booth or of untappable
channels.

Receipt-freeness is a very subtle property. The problem is that the voter
{\em wants\/} to prove his vote, and hence he might even deviate from the
protocol in order to construct a receipt. Even worse, there are no sharp
definitions of receipt-freeness, and many different flavors of this
property are implicitly considered in the literature. Several of the
proposed protocols were partially or completely broken
\cite{MicHor96,Okamot97,Schoen99:pc,HirSak00}.


\desc

The goal of this work is to bring light into the definitions and models of
receipt-freeness. So far, many different (and incompatible) notions are
used, and comparison of proposed protocols and models is very difficult.
Several conjectures and propositions on impossibility or optimality of
certain models can be found in the literature, where formal definitions and
proofs are missing. Special focus should be given to practical models,
leading to the question what type of receipt-freeness is still possible
with realistic assumptions.


\tasks

The following is an (incomplete) list of tasks:
\begin{itemize}
  \item Study of the literature,
  \item find exact definitions of receipt-freeness,
  \item propose realistic models for receipt-free voting protocols,
  \item proof sufficiency and necessity of assumptions.
\end{itemize}

\blah

\refs

\bibliography{mpc,ba,complex,it,ss,voting,proof,classic,all,allno,allspec}

\end{document}
