\documentclass{article}

\title{\bf Efficient Receipt-Free Voting\\
       Based on Homomorphic Encryption}
\author{Martin Hirt \and Kazue Sako}


\begin{document}

\maketitle

\begin{abstract}
  Voting schemes that provide {\em receipt-freeness\/} prevent voters from
  proving their cast vote, and hence thwart vote-buying and
  coercion. 
  We analyze the security of the multi-authority voting
  protocol of Benaloh and Tuinstra and demonstrate that this protocol is
  {\em not\/} receipt-free, opposed to what was claimed in the paper and
  was believed before. Furthermore, we propose the first practicable
  receipt-free voting scheme. Its only physical assumption is the existence
  of secret {\em one-way\/} communication channels from the
  authorities to the voters, and due to the public verifiability of the
  tally, voters only join a single stage of the protocol, realizing the
  ``vote-and-go'' concept. The protocol combines the advantages of the
  receipt-free protocol of Sako and Kilian and of the very efficient
  protocol of Cramer, Gennaro, and Schoenmakers, with help of
  designated-verifier proofs of Jakobsson, Sako, and Impagliazzo.

  Compared to the receipt-free protocol of Sako and Kilian for security
  parameter $\ell$ (the number of repetitions in the non-interactive
  cut-and-choose proofs), the protocol described in this paper realizes an
  improvement of the total bit complexity by a factor $\ell$.

  \smallskip\noindent{\bf Key words.}
  Receipt-free voting, Uncoercibility, Designated-verifier proofs.
\end{abstract}


%----------------------------------------------------------------------

\section{Introduction}\label{intro}

\subsection{Background}

Secret-ballot voting protocols are one of the most significant
application of cryptographic protocols. The most efficient
secret-ballot voting protocols can be categorized by their approaches
into three types: Schemes using mix-nets
\cite{Cha81,PIK93,SK95,OKS97,Jak98,Abe99}, schemes using homomorphic
encryption \cite{CF85,CY86,Ben87,%Ive91,
BT94,SK94,CFSY96,CGS97}, and schemes using
blind signatures \cite{FOO92,Sak94,Oka97}. The suitability of each of
these three types varies with the conditions under which it is to be
applied.

In a model with vote-buyers (or coercers), a voting scheme must ensure not
only that a voter {\em can\/} keep his vote private, but also that he
{\em must\/} keep it private. In other words, the voter should not be
able to prove to a third party that he has cast a particular vote. He must
neither obtain nor be able to construct a receipt proving the content of
his vote. This property is referred to as {\em receipt-freeness\/}.

The concept of receipt-freeness was first introduced by Benaloh and
Tuinstra \cite{BT94}. Based on the assumption of a voting booth that
physically guarantees
%two-way
secret communication between the authorities and each voter,
they proposed two voting protocols using homomorphic
encryptions.  The first one is a single-authority voting protocol which,
while being receipt-free, fails to maintain vote secrecy.  Then they extend
this protocol to the second protocol, which is a multi-authority scheme
achieving vote secrecy.  However, we show that this scheme is {\em not\/}
receipt-free, as opposed to what is claimed in the paper.

Another receipt-free voting protocol based on a mix-net channel was
proposed by Sako and Kilian \cite{SK95}. In contrast to \cite{BT94}, it
assumes only {\em one-way\/} secret communication from the authorities to
the voters. The heavy processing load required for tallying in mix-net
schemes, however, is a significant disadvantage of this protocol.

Finally, a receipt-free voting scheme using blind signatures was given by
Okamoto \cite{Oka97}. Here, the assumption was of {\em anonymous\/} one-way
secret communication from each voter to each authority. Achieving
communication that is both secret {\em and\/} anonymous would, however, be
extremely difficult. Also, this scheme requires each voter to be active in
{\em three rounds\/} (authorization stage, voting stage, and claiming
stage), which is not acceptable in practice.

Another stream of research which relates to receipt-freeness is incoercible
multi-party computation. Without any physical assumption, deniable
encryption \cite{CDNO97} allows an entity to lie {\em later\/} how the
ciphertext decrypts, and this technique is used to achieve incoercible
multi-party computation \cite{CG96}. However, the concept of incoercibility
is weaker than receipt-freeness. It would allow a voter to lie about his
vote, but it cannot help against a voter who {\em wants\/} to make his
encryption undeniable, and hence cannot prevent vote-buying.


\subsection{Contributions}

In this paper, we first demonstrate that the multi-authority protocol
of Benaloh and Tuinstra \cite{BT94} is not receipt-free, opposed to what
was claimed in the paper and was believed before.
%\footnote{The
%  single-authority protocol of \cite{BT94}, which is later extended to the
%  multi-authority protocol, indeed seems to be receipt-free, but
%  ``elections where one authority knows how each vote was cast are quite
%  far from satisfactory'' \cite{BT94}.}
We then present a novel generic construction for introducing
receipt-freeness into a voting scheme based on homomorphic encryption
by assuming some additional properties of the encryption function.
This construction also includes a solution for the case that an
authority does not send correct information through the untappable
channel.\footnote{Due to the untappability of the channel the voter
  cannot prove that the received information is incorrect. In previous
  protocols, this problem was ignored, and the situation of a voter
  complaining about an authority would have lead to a deadlock.}
Moreover, as opposed to previous receipt-free protocols, we disable
vote-buying even in cases where some authorities are colluding with
the voter-buyer. The security of these protocols is specified with
respect to a threshold $t$, where the correctness of the tally is
guaranteed as long as at least $t$ authorities remain honest during
the whole protocol execution, and privacy is guaranteed as long as no
$t$ or more curious authorities pool their information.

Our construction gives a receipt-free voting protocol which runs as
follows: For each voter the authorities jointly generate a randomly
ordered list with an encryption of each valid vote, along the lines
of \cite{SK95}. The ordering of the list is secretly conveyed and proven
to the voter by deploying the technique of designated-verifier proofs
\cite{JSI96}, and the voter points to the encryption of his
choice. Tallying of votes is performed using the homomorphic property
of the encryption function.

By applying this generic construction to the voting protocol of
Cramer, Gennaro, and Schoenmakers \cite{CGS97}, we obtain an efficient
receipt-free voting protocol based on homomorphic encryption.

The efficiency achieved by our protocol compared to the protocol
of Sako and Kilian \cite{SK95} with security parameter $\ell$ in the
case of $1$-out-of-$2$ voting is as follows:
The communication through the untappable channels and through the
public channels are reduced by a factor of $\ell/4$ and $3\ell/2$,
respectively. Altogether, this results in a speedup by a factor of $\ell$.
As an example, for $M=1,000,000$ voters, $N=10$ authorities, a $K=1024$ bit
group, and security parameter $\ell=80$, the protocol of \cite{SK95}
communicates $102\,\mathrm{GB}$ (gigabyte) over the untappable channels and
$924\,\mathrm{GB}$ over the public channels, whereas the protocol of this
paper communicates $5\,\mathrm{GB}$ over untappable channels and
$8\,\mathrm{GB}$ over public channels.

\subsection{Organization of the Paper}

The paper is organized as follows: In Sect.~\ref{sec:BT}, we analyze the
receipt-freeness of the protocol with multiple voting authorities of
Benaloh and Tuinstra \cite{BT94} and demonstrate its non receipt-freeness
by showing how a voter can construct a receipt for the vote he casts. In
Sect.~\ref{sec:genconstr}, we present a generic receipt-free protocol
for $1$-out-of-$L$ voting based on homomorphic encryptions, and in
Sect.~\ref{sec:protCGS} we apply these techniques to the protocol of
Cramer, Gennaro, and Schoenmakers \cite{CGS97} and obtain an efficient
receipt-free voting scheme. Finally, in Sect.~\ref{sec:eff}, we even
improve the efficiency of our protocol by tailoring it to $1$-out-of-$2$
voting.


%----------------------------------------------------------------------

\section{Analysis of the Benaloh-Tuinstra Protocol}\label{sec:BT}

The notion of receipt-freeness was first introduced by Benaloh and Tuinstra
in \cite{BT94}.  They present two protocols that are claimed to be
receipt-free. In the single-authority protocol, the authority learns how
each vote was cast. This is of course far from satisfactory.  In this
section, we analyze the receipt-freeness of their protocol with multiple
voting authorities and show how a voter can construct a receipt for the
vote he casts.


\subsection{Key Ideas of the Protocol}

The basic idea of the multiple-authority protocol \cite{BT94} is to have
every voter secret-share his vote among the authorities (using Shamir's
secret-sharing scheme \cite{Sha79}), who then add up the shares and
interpolate the tally. This idea works due to the linearity of the
secret-sharing scheme. There are two major tasks to solve: First, the
voter must send one share to each authority in a receipt-free manner,
and second, the voter must prove that the secret (the vote) is valid.

We concentrate on the second task: In order to secret-share the vote,
the voter selects a random polynomial $P$ of appropriate degree, such
that $P(0)\in\{0,1\}$ is his vote. The share for the $j$-th authority is
hence $P(j)$. Clearly, it is inherently important that the vote is
valid, i.e. $P(0)\in\{0,1\}$, since otherwise the tally will be
incorrect. Hence, the voter must provide a proof of validity for the
cast vote.

For the sake of the proof of validity, the voter wishing to cast a vote
$v_0$ submits a bunch of $n+1$ vote pairs, where $n$ is a security
parameter. That is, the voter submits the votes $(v_0,v_0'), \ldots,
(v_n,v_n')$, and each pair $(v_i,v_i')$ of votes must contain one $0$-vote
and one $1$-vote in random order.
For each pair $(v_i,v_i')$ but the first, a coin is tossed and the voter
is either asked to open the pair and show that indeed there is a
$0$-vote and a $1$-vote, or he is asked to prove that either $v_i=v_0$
and $v_i'=v_0'$ is satisfied, or that $v_i=v_0'$ and $v_i'=v_0$ is
satisfied. If the voter passes these tests, then with probability at
least $1-2^{-n}$, $v_0$ is valid and is accepted as the voters vote.

\subsection{How to Construct a Receipt}

This cut-and-choose proof of validity offers an easy ability to prove a
particular vote: In advance, the voter commits to the ordering of each
pair of votes (i.e.~he commits to the bit string $v_0,\ldots,v_n$). In
each round of the cut-and-choose proof, one can verify whether the
revealed data is consistent with this commitment. If no inconsistencies
are detected while proving the validity of the vote, then with
probability at least $1-2^{-n}$ the voter has chosen the ordering as
committed, and also $v_0$ is as announced.

In order to obtain a receipt, the voter could select an arbitrary string
$s$, and set the string $(v_0,\ldots,v_n)$ as the bitwise output of a
known cryptographic hash function (e.g. MD5 or SHA) for that string
$s$. Then, $s$ is a receipt of the vote $v_0$.


%----------------------------------------------------------------------

\section{Generic Receipt-Free Protocol}\label{sec:genconstr}

\ldots


\section{Made Receipt-Free}

\section{Efficient Receipt-Free 1-out-of-2 Voting}
\label{sec:eff}


\section{Concluding Remarks}

\section{Acknowledgments}

\begin{appendix}
\section{Ensuring Knowledge of the Secret-Key}
\label{app:proveSK}

In a model providing receipt-freeness, it is essential that each voter
knows his own secret-key. We assume that this verification is part of the
underlying public-key infrastructure, but nevertheless we provide a
protocol that ensures a voter's  knowledge of his secret-key. This
protocol may be performed as part of the key registration (in the
public-key infrastructure), or as part of the voting protocol if the key
infrastructure does not provide this property. This protocol requires a
secure one-way untappable channel as used in the vote generation phase.

The following protocol is based on Feldman's secret-sharing scheme
\cite{Fel87}. It establishes that a voter $v$ knows the secret key
$z_v$ corresponding to his public key $h_v$ (where $g^{z_v}=h_v$):

\begin{itemize}
\item The voter shares his secret key $z_v$ among the authorities by using
  Feldman's secret-sharing scheme \cite{Fel87}: The voter $v$ chooses a
  uniformly distributed random polynomial
  $f_v(x)=z_v+a_1x+\ldots+a_{t-1}x^{t-1}$ of degree $t-1$, and secretly
  sends\footnote{Either the voter encrypts the share with the authority's
    public-key, or alternatively the authority first sends a one-time pad
    through the untappable channel, and the voter then encrypts with this
    pad.} the share $s_i=f_v(i)$ to authority $A_i$ (for $i=1,\ldots,N$).
  Furthermore, the voter commits to the coefficient of the polynomial by
  sending $c_i=g^{a_i}$ for $i=1,\ldots,t-1$ to the bulletin board.
\item Each authority $A_i$ verifies with the following equation whether the
  received share $s_i$ indeed lies on the committed polynomial $f_v(\cdot)$:
  \[ g^{s_i} \stackrel{?}{=}
       h_v \cdot c_1^i \cdot \ldots \cdot c_{t-1}^{i^{t-1}}
     \ \ \ \ 
     \Bigl( = g^{z_v} \cdot g^{a_1i} \cdots \ldots \cdots g^{a_{t-1}i^{t-1}}
            = g^{f_v(i)} \Bigr)
     \ .
  \]
  If an authority detects an error, she complains and the voter is
  requested to post her share to the bulletin board. If the posted share
  does not correspond to the commitments, the voter is disqualified.
\item Finally, every authority (which did not complain in the previous
  stage) sends her share through the untappable channel to the voter.
\end{itemize}

In the above protocol, clearly after the second step, either the (honest)
authorities will have consistent shares of the voter's secret key $z_v$, or
the voter will be disqualified. However, so far it is not ensured that the
voter indeed knows the secret key, as the shares could have been provided
by the coercer. In any case, in the final step the voter learns
$z_v$. There are at least $t$ honest authority who either complained 
(and thus their share is published), or who sent their share to the voter,
and hence the voter can interpolate the secret key $z_v$.

\end{appendix}

%----------------------------------------------------------------------

\begin{thebibliography}{[CDNO97]}

\bibitem[Abe99]{Abe99}
Masayuki Abe.
%M. Abe.
\newblock Mix-networks on permutation networks
\newblock In {\em Advances in Cryptology --- {ASIACRYPT}~'99},
  vol.~1716 of {\em LNCS}, pp.~258--273.
  {Springer-Verlag}, 1999.

\bibitem[Ben87]{Ben87}
Josh~Cohen Benaloh:
\newblock {\em Verifiable Secret-Ballot Elections.}
\newblock Yale University PhD thesis, YALEU/DCS/TR-561, 1987.

\bibitem[BT94]{BT94}
Josh~Cohen Benaloh and Dwight Tuinstra.
%J. Benaloh and D. Tuinstra.
\newblock Receipt-free secret-ballot elections (extended abstract).
\newblock In {\em Proc.~26th {ACM} Symposium on the Theory of Computing
  ({STOC})}, pp.~544--553.
  ACM, 1994.

\bibitem[CDNO97]{CDNO97}
Ran Canetti, Cynthia Dwork, Moni Naor, and Rafail Ostrovsky.
%R. Canetti, C. Dwork, M. Naor, and R. Ostrovsky.
\newblock Deniable encryption.
\newblock In {\em Advances in Cryptology --- {CRYPTO}~'97},
  vol.~1294 of {\em LNCS}, pp.~90--104.
  {Springer-Verlag}, 1997.

\bibitem[CG96]{CG96}
Ran Canetti and Rosario Gennaro.
%R. Canetti and R. Gennaro.
\newblock Incoercible multiparty computation.
\newblock In {\em Proc.~37th {IEEE} Symposium on the Foundations of 
  Computer Science ({FOCS})}, 1996.

\bibitem[Cha81]{Cha81}
David Chaum.
%D. Chaum.
\newblock Untraceable electronic mail, return addresses, and digital
  pseudonyms.
\newblock {\em Communications of the ACM}, 24(2):84--88, 1981.

\bibitem[CDS94]{CDS94}
Ronald Cramer, Ivan Damg{\aa}rd, and Berry Schoenmakers.
%R. Cramer, I. Damg{\aa}rd, and B. Schoenmakers.
\newblock Proofs of partial knowledge and simplified design of witness 
  hiding protocols.
\newblock In {\em Advances in Cryptology --- {CRYPTO}~'94},
  vol.~839 of {\em LNCS}.
  {Springer-Verlag}, 1994.

\bibitem[CF85]{CF85}
Josh~D. Cohen (Benaloh) and Michael~J. Fischer.
%J. Cohen (Benaloh) and M. Fischer.
\newblock A robust and verifiable cryptographically secure election
  scheme.
\newblock In {\em Proc.~26th {IEEE} Symposium on the Foundations of 
  Computer Science ({FOCS})}, pp.~372--382. IEEE, 1985.

\bibitem[CY86]{CY86}
Josh Cohen (Benaloh) and Moti Yung:
\newblock Distributing the Power of a Government to Enhance the Privacy of
  Voters.
\newblock In {\em Proc.~5th {ACM} Symposium on Principles of Distributed
  Computing ({PODC})}, pp.~52--62.
  ACM, 1986.

\bibitem[CFSY96]{CFSY96}
Ronald Cramer, Matthew~K. Franklin, Berry Schoenmakers, and Moti Yung.
%R. Cramer, M. Franklin, B. Schoenmakers, and M. Yung.
\newblock Multi-authority secret-ballot elections with linear work.
\newblock In {\em Advances in Cryptology --- {EUROCRYPT}~'96},
  vol.~1070 of {\em LNCS}, pp.~72--83.
  {Springer-Verlag}, May 1996.

\bibitem[CGS97]{CGS97}
Ronald Cramer, Rosario Gennaro, and Berry Schoenmakers.
%R. Cramer, R. Gennaro, and B. Schoenmakers.
\newblock A secure and optimally efficient multi-authority election
  scheme.
\newblock {\em European Transactions on Telecommunications}, 8:481--489,
  1997.
  Preliminary version in {\em Advances in Cryptology --- {EUROCRYPT}~'97}.

\bibitem[E84]{E84}
Taher ElGamal.
%T. ElGamal.
\newblock A public key cryptosystem and a signature scheme based on 
  discrete logarithms.
\newblock In {\em Advances in Cryptology --- {CRYPTO}~'84},
  vol.~196 of {\em LNCS}, pp.~10--18.
  {Springer-Verlag}, 1984.

\bibitem[Fel87]{Fel87}
Paul Feldman.
%P. Feldman.
\newblock A practical scheme for non-interactive verifiable secret sharing.
\newblock In {\em Proc.~28th {IEEE} Symposium on the Foundations of Computer
  Science ({FOCS})}, pp.~427--437, 1987.

\bibitem[FS86]{FS86}
Amos Fiat and Adi Shamir.
%A. Fiat and A. Shamir.
\newblock How to prove yourself: Practical solutions to identification
  and signature problems.
\newblock In {\em Advances in Cryptology --- {CRYPTO}~'86},
  vol.~263 of {\em LNCS}, pp.~186--194.
  {Springer-Verlag}, 1986.

\bibitem[FOO92]{FOO92}
Atsushi Fujioka, Tatsuaki Okamoto, and Kazuo Ohta.
%A. Fujioka, T. Okamoto, and K. Ohta.
\newblock A practical secret voting scheme for large scale elections.
\newblock In {\em Advances in Cryptology --- {AUSCRYPT}~'92},
  pp.~244--251, 1992.

\bibitem[JSI96]{JSI96}
Markus Jakobsson, Kazue Sako, and Russell Impagliazzo.
%M. Jakobsson, K. Sako, and R. Impagliazzo.
\newblock Designated-verifier proofs and their applications.
\newblock In {\em Advances in Cryptology --- {EUROCRYPT}~'96},
  vol.~1070 of {\em LNCS}, pp.~143--154.
  {Springer-Verlag}, 1996.

\bibitem[Jak98]{Jak98}
Markus Jakobsson.
%M. Jakobsson.
\newblock A Practical Mix.
\newblock In {\em Advances in Cryptology --- {EUROCRYPT}~'98},
  vol.~1403 of {\em LNCS}, pp.~448--461,
  {Springer-Verlag}, 1998.

\bibitem[MH96]{MH96}
Markus Michels and Patrick Horster.
%M. Michels and P. Horster.
\newblock Some remarks on a receipt-free and universally verifiable
  mix-type voting scheme.
\newblock In {\em Advances in Cryptology --- {ASIACRYPT}~'96},
  vol.~1163 of {\em LNCS}, pp.~125--132.
  {Springer-Verlag}, 1996.

\bibitem[Oka97]{Oka97}
Tatsuaki Okamoto.
%T. Okamoto.
\newblock Receipt-free electronic voting schemes for large scale
  elections.
\newblock In {\em Proc.~of Workshop on Security Protocols~'97},
  vol.~1361 of {\em LNCS}, pp.~25--35.
  {Springer-Verlag}, 1997.

\bibitem[OKST97]{OKS97}
Wakaha Ogata, Kaoru Kurosawa, Kazue Sako and Kazunori Takatani
%W. Ogata, K. Kurosawa, K. Sako and K. Takatani
\newblock Fault Tolerant Anonymous Channel
\newblock In {\em Information and Communications Security {ICICS}~'97},
  vol.~1334 of {\em LNCS}, pp.~440--444.
  {Springer-Verlag}, 1997.

\bibitem[PIK93]{PIK93}
Choonsik Park, Kazutomo Itoh, and Kaoru Kurosawa.
%C. Park, K. Itoh, and K. Kurosawa.
\newblock Efficient anonymous channel and all/nothing election scheme.
\newblock In {\em Advances in Cryptology --- {EUROCRYPT}~'93},
  vol.~765 of {\em LNCS}, pp.~248--259.
  {Springer-Verlag}, 1993.

\bibitem[Ped91]{Ped91}
Torben~P. Pedersen.
%T. Pedersen.
\newblock A threshold cryptosystem without a trusted party (extended
  abstract).
\newblock In {\em Advances in Cryptology --- {EUROCRYPT}~'91},
  vol.~547 of {\em LNCS}, pp.~522--526.
  {Springer-Verlag}, 1991.

\bibitem[Sak94]{Sak94}
Kazue Sako.
%K. Sako.
\newblock Electronic voting schemes allowing open objection to the tally.
\newblock In {\em Transactions of IEICE,} vol.~E77-A No.1, Jan.
  1994.

\bibitem[SK94]{SK94}
Kazue Sako and Joe Kilian.
%K. Sako and J. Kilian.
\newblock Secure voting using partially compatible homomorphisms.
\newblock In {\em Advances in Cryptology --- {CRYPTO}~'94}, vol.~839
  of {\em LNCS}, pp.~411--424.
  {Springer-Verlag}, 1994.

\bibitem[SK95]{SK95}
Kazue Sako and Joe Kilian.
%K. Sako and J. Kilian.
\newblock Receipt-free mix-type voting scheme -- {A} practical solution
  to the implementation of a voting booth.
\newblock In {\em Advances in Cryptology --- {EUROCRYPT}~'95},
  vol.~921 of {\em LNCS}, pp.~393--403.
  {Springer-Verlag}, 1995.

\bibitem[Sha79]{Sha79}
Adi Shamir.
%A. Shamir.
\newblock How to share a secret.
\newblock {\em Communications of the ACM}, 22:612--613, 1979.

\end{thebibliography}

\end{document}
